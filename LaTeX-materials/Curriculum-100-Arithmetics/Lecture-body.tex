\begin{document}
\providecommand{\currentLecture}{6}


\lect{\semester}{Lecture 1: Addition}{1}{
%DesiredLectureName: Addition
\section{Addition}
\input{../../freecalc/modules/large-integers/integer-addition-decimal-one-digit-addition-example-1}

\input{../../freecalc/modules/large-integers/integer-addition-decimal-one-digit-addition-with-table-example-1}

\input{../../freecalc/modules/large-integers/integer-addition-decimal-one-digit-addition-with-table-columns-example-1} 


\input{../../freecalc/modules/large-integers/integer-addition-decimal-example-1}
\input{../../freecalc/modules/large-integers/integer-addition-decimal-example-2}
\input{../../freecalc/modules/large-integers/integer-addition-decimal-example-3}
\input{../../freecalc/modules/large-integers/integer-addition-decimal-example-4}
\input{../../freecalc/modules/large-integers/integer-addition-algorithm}
}

\lect{\semester}{Lecture 2: Subtraction}{2}{
%DesiredLectureName: Subtraction
\section{Subtraction}
\input{../../freecalc/modules/large-integers/integer-subtraction-one-digit-positive-result-example-1}

\input{../../freecalc/modules/large-integers/integer-subtraction-one-digit-positive-result-with-table-example-1}

\input{../../freecalc/modules/large-integers/integer-subtraction-one-digit-positive-result-with-table-columns-example-1}
\input{../../freecalc/modules/large-integers/integer-subtraction-two-digit-minus-one-digit-result-one-digit-positive-columns-example-1}

\input{../../freecalc/modules/large-integers/negative-integers-1}
\input{../../freecalc/modules/large-integers/integer-subtraction-algebra-minus-sign-as-operator-1}
\input{../../freecalc/modules/large-integers/integer-subtraction-algebra-magnitude-of-a-number-1}

\input{../../freecalc/modules/large-integers/integer-subtraction-algebra-negative-of-a-negative-1}

\input{../../freecalc/modules/large-integers/integer-subtraction-algebra-sum-with-negative-1}
\input{../../freecalc/modules/large-integers/integer-subtraction-algebra-negative-of-sum-1}

\input{../../freecalc/modules/large-integers/integer-subtraction-algebra-rules-summary-1}

\input{../../freecalc/modules/large-integers/integer-subtraction-equation-a-plus-x-equals-b-example-with-theory-1}

\input{../../freecalc/modules/large-integers/integer-subtraction-equation-a-plus-x-equals-b-answer-negative-1}
\input{../../freecalc/modules/large-integers/integer-subtraction-equation-a-plus-x-equals-b-1}

\input{../../freecalc/modules/large-integers/integer-subtraction-decimal-represent-minus-one-digit-as-ten-plus-digit-1}

\input{../../freecalc/modules/large-integers/integer-subtraction-decimal-two-digit-minus-one-digit-example-1}
\input{../../freecalc/modules/large-integers/integer-subtraction-decimal-full-algorithm-example-1}
\input{../../freecalc/modules/large-integers/integer-subtraction-decimal-full-algorithm-example-2}
\input{../../freecalc/modules/large-integers/integer-subtraction-decimal-full-algorithm-example-large-1}
}

\lect{\semester}{Lecture 3: Multiplication}{3}{
%DesiredLectureName: Multiplication
\section{Multiplication}
\subsection{Multiplication of one-digit numbers}
\input{../../freecalc/modules/large-integers/integer-multiplication-decimal-one-digit-multiplication-no-table-example-1}
\input{../../freecalc/modules/large-integers/integer-multiplication-decimal-one-digit-multiplication-with-table-example-1}

\subsection{Multiplication of one-digit number by multi-digit number}
\input{../../freecalc/modules/large-integers/integer-multiplication-by-power-of-base-10-theory-and-example-1}

\input{../../freecalc/modules/large-integers/integer-multiplication-decimal-example-one-digit-by-larger-no-carry-over-1}
\input{../../freecalc/modules/large-integers/integer-multiplication-decimal-example-one-digit-by-larger-with-carry-over-1}

\subsection{Multiplication of multi-digit numbers}
\input{../../freecalc/modules/large-integers/integer-multiplication-decimal-example-1}
\input{../../freecalc/modules/large-integers/integer-multiplication-decimal-example-2}
\input{../../freecalc/modules/large-integers/integer-multiplication-decimal-example-3}
\input{../../freecalc/modules/large-integers/integer-multiplication-decimal-example-4}
\input{../../freecalc/modules/large-integers/integer-multiplication-decimal-example-5}
}% end lecture

\lect{\semester}{Lecture 4: Fractions}{4}{
%DesiredLectureName: Fraction basics
\section{Fraction basics}
\subsection{Fractions definition}
\input{../../freecalc/modules/division/fractions-and-the-number-line}
\input{../../freecalc/modules/division/fractions-denominator-one}
\input{../../freecalc/modules/division/fractions-reading-1}
\subsection{Add fractions with same denominator}
\input{../../freecalc/modules/division/fractions-rules-add-same-denominator}
\input{../../freecalc/modules/division/fractions-add-same-denominator-exercise-1}

\subsection{Number factorization and prime numbers}
\input{../../freecalc/modules/division/factor-a-number}
\input{../../freecalc/modules/division/prime-numbers}
\input{../../freecalc/modules/division/factorization}
\input{../../freecalc/modules/division/factorization-small-example-1}
\input{../../freecalc/modules/division/factorization-terminology}

\input{../../freecalc/modules/division/factorization-algorithmic-difficulty-1}


\subsection{Number factorization and fraction reduction}
\input{../../freecalc/modules/division/fractions-rules-scaling}
\input{../../freecalc/modules/division/fractions-reduced}

\input{../../freecalc/modules/division/fractions-cancel-notation}
\input{../../freecalc/modules/division/fractions-simplify-example-1}


\input{../../freecalc/modules/division/fractions-simplify-algorithm-promise}
\subsection{Multiply fraction by integer}
\input{../../freecalc/modules/division/fractions-multiplication-by-integer}
\input{../../freecalc/modules/division/fractions-multiplication-by-integer-example-1}
}

\lect{\semester}{Lecture 5: Division}{5}{
%DesiredLectureName: Division
\section{Division and fractions}
\input{../../freecalc/modules/division/division-and-fractions}
\input{../../freecalc/modules/division/division-with-remainder-definitions}
\input{../../freecalc/modules/division/division-with-remainder-interpretation}
\input{../../freecalc/modules/division/division-with-remainder-addition-notation-exercise-with-guidance-question-1}
\input{../../freecalc/modules/division/division-with-remainder-addition-notation-one-digit-divisor-one-digit-dividend-1}
\input{../../freecalc/modules/division/division-with-remainder-addition-notation-exercise-one-digit-divisor-one-digit-quotient-1}

\input{../../freecalc/modules/continuity/greatest-integer-function-rationals-no-negatives}
\input{../../freecalc/modules/continuity/greatest-integer-function-some-rules-with-example}
\input{../../freecalc/modules/division/division-with-remainder-integer-part-of-fraction}

\input{../../freecalc/modules/division/division-with-remainder-integer-part-of-fraction-exercise-1}
}

\lect{\semester}{Lecture 6: Long division}{6}{
\section{Long division}

\input{../../freecalc/modules/division/integer-long-division-intro-1}

\input{../../freecalc/modules/division/integer-long-division-one-digit-one-step-example-1}
\input{../../freecalc/modules/division/integer-long-division-intro-2}
\input{../../freecalc/modules/division/integer-long-division-one-digit-multiple-step-example-1}
\input{../../freecalc/modules/division/integer-long-division-one-digit-multiple-step-example-2}
\input{../../freecalc/modules/division/integer-long-division-one-digit-multiple-step-example-3}
\input{../../freecalc/modules/division/integer-long-division-intro-3}
\input{../../freecalc/modules/division/integer-long-division-multi-digit-divisor-example-1}
\input{../../freecalc/modules/division/integer-long-division-multi-digit-divisor-example-2}
\input{../../freecalc/modules/division/integer-long-division-multi-digit-divisor-example-3}
\input{../../freecalc/modules/division/integer-long-division-multi-digit-divisor-example-4}
\input{../../freecalc/modules/division/integer-long-division-multi-digit-divisor-example-5}
\input{../../freecalc/modules/division/integer-long-division-multi-digit-divisor-example-6}
\input{../../freecalc/modules/division/integer-long-division-multi-digit-divisor-example-7}
\input{../../freecalc/modules/division/integer-long-division-how-many-quotient-rows-1}

\input{../../freecalc/modules/division/integer-long-division-multi-digit-divisor-example-8}


\input{../../freecalc/modules/division/integer-long-division-knuth-optimization-1}
}

\lect{\semester}{Lecture 7: Floating point}{7}{
\input{../../freecalc/modules/series/series-geometric-ex1.tex}
\input{../../freecalc/modules/series/series-geometric-theorem.tex}
\input{../../freecalc/modules/series/series-geometric-example-0.tex}
\input{../../freecalc/modules/series/series-geometric-repeating-decimal-as-rational-example-1.tex}
\input{../../freecalc/modules/series/series-rational-to-repeating-decimal-algorithm.tex}
\input{../../freecalc/modules/series/series-rational-to-repeating-decimal-example-1.tex}
\input{../../freecalc/modules/series/series-rational-to-repeating-decimal-example-2.tex}
}

\lect{\semester}{Lecture 8}{8}{
\section{More}
\input{../../freecalc/modules/large-integers/integer-base}
\input{../../freecalc/modules/large-integers/integer-hexadecimal}
}

\end{document}
