\documentclass
%[handout]
{beamer}
\newcommand{\semester}{2018}
\usepackage[breakwords]{truncate}
\input{../../freecalc/lectures/example-templates}
\input{../../freecalc/lectures/system-specific-config-Ubuntu-texlive} %
\input{../../freecalc/lectures/pstricks-commands}
\renewcommand{\Arcsin}{\arcsin}
\renewcommand{\Arccos}{\arccos}
\renewcommand{\Arctan}{\arctan}
\renewcommand{\Arccot}{\text{arccot\hspace{0.03cm}}}
\renewcommand{\Arcsec}{\text{arcsec\hspace{0.03cm}}}
\renewcommand{\Arccsc}{\text{arccsc\hspace{0.03cm}}}
\useinnertheme{rounded}
\useoutertheme{infolines}
\usecolortheme{orchid}
\usecolortheme{whale}


\usepackage[T1]{fontenc}
% Or whatever. Note that the encoding and the font should match. If T1
% does not look nice, try deleting the line with the fontenc.

\setbeamertemplate{navigation symbols}{}

\newcommand{\lect}[4]{
\ifnum#3=\currentLecture
  \date{#1}
  \lecture[#1]{#2}{#3}
#4
\else
%include nothing
\fi
}

\setbeamertemplate{headline}
{
  \leavevmode%
  \hbox{%
  \begin{beamercolorbox}[wd=.4\paperwidth,ht=2.25ex,dp=1ex,center]{author in head/foot}%
    \usebeamerfont{title in head/foot}{\insertsectionhead}
  \end{beamercolorbox}%
  \begin{beamercolorbox}[wd=.4\paperwidth,ht=2.25ex,dp=1ex,center]{title in head/foot}%
    \usebeamerfont{title in head/foot}{\insertsubsectionhead}
  \end{beamercolorbox}%
\begin{beamercolorbox}[wd=.2\paperwidth,ht=2.25ex,dp=1ex,center]{date in head/foot}%
\insertframenumber/\inserttotalframenumber
  \end{beamercolorbox}
}
}

\setbeamertemplate{footline}
{
  \leavevmode%
  \hbox{%
  \begin{beamercolorbox}[wd=.4\paperwidth,ht=2.25ex,dp=1ex,center]{author in head/foot}%
    \usebeamerfont{author in head/foot}\insertshortauthor
  \end{beamercolorbox}%
  \begin{beamercolorbox}[wd=.4\paperwidth,ht=2.25ex,dp=1ex,center]{title in head/foot}%
    \usebeamerfont{title in head/foot}{\insertshorttitle}
  \end{beamercolorbox}%
  \begin{beamercolorbox}[wd=.2\paperwidth,ht=2.25ex,dp=1ex,center]{date in head/foot}%
    \usebeamerfont{date in head/foot}\insertshortdate{}
  \end{beamercolorbox}}%
  \vskip0pt%
}


\AtBeginLecture{%
\title[{\truncate{4.2cm}{\insertlecture}}]{
\textbf{\color{green}{Cryptography 101}}
\\
\textbf{\insertlecture} \\
{\color{green}{\textbf{calculator-algebra.org}}}
}
\author[{\color{green}{\bf calculator-algebra.org}}, Todor Milev]{
~\\~\\~\\
\begin{tabular}{c}
\Large Todor Milev 
\end{tabular}
}


\date{\insertshortlecture}

\begin{frame}
\titlepage
\end{frame}
}

\begin{document}
\providecommand{\currentLecture}{8}

\lect{\semester}{Angles}{1}{% begin lecture
%DesiredLectureName: Angles
\section{Angles}
\subsection{The Unit circle}
\input{../../modules/trigonometry/unit-circle-def}
\subsection{Three Meanings of Angle}
\input{../../modules/trigonometry/angle-three-meanings}
\input{../../modules/trigonometry/angle-geometric-definition}
\input{../../modules/trigonometry/angle-measure-of-geometric-angle-definition}
\input{../../modules/trigonometry/circle-arclength-from-general-definition-note}
\input{../../modules/trigonometry/circle-arclength}
\input{../../modules/trigonometry/circle-arclength-ex1}
\subsection{Two Meanings of Rotation}
\input{../../modules/trigonometry/rotation-two-meanings}
\input{../../modules/trigonometry/continuous-rotation-definition}
\input{../../modules/trigonometry/angle-measure-of-continuous-rotation}
\input{../../modules/trigonometry/angle-equivalence}
\subsection{Angles and the Coordinate System}
\input{../../modules/trigonometry/angle-coordinate-system}
\subsection{Radians and Degrees}
\input{../../modules/trigonometry/angles-measurement-units}
\input{../../modules/trigonometry/convert-radians-degrees-ex1}
\input{../../modules/trigonometry/circle-arclength-ex2}
\input{../../modules/trigonometry/frequently-encountered-angles}
\input{../../freecalc/modules/trigonometry/coterminal-angles}
\input{../../modules/trigonometry/coterminal-angles-example-1}
\input{../../modules/trigonometry/complementary-angles}
\input{../../modules/trigonometry/supplementary-angles}
\subsection{Area cut off by an angle}
\input{../../modules/trigonometry/area-sector}
}

\lect{\semester}{Lecture 2}{2}{% begin lecture
%DesiredLectureName: Trigonometry_Definitions
\section{Trigonometry}
\subsection{Definition of the Trigonometric Functions}
\input{../../freecalc/modules/trigonometry/trig-functions-definition}
\input{../../freecalc/modules/trigonometry/similar-triangles-def}
\input{../../freecalc/modules/trigonometry/similar-triangles-same-side-ratios}
\input{../../freecalc/modules/trigonometry/trig-functions-and-right-angle-triangles}
\subsection{Basic Computations with Trigonometric Functions}
\input{../../freecalc/modules/trigonometry/trig-functions-example3}
\input{../../freecalc/modules/trigonometry/trig-functions-example2}
\input{../../freecalc/modules/trigonometry/triangle-angles-sum-to-180-degrees}
\input{../../freecalc/modules/trigonometry/trig-functions-of-45-degrees}
\input{../../freecalc/modules/trigonometry/trig-functions-of-30-60-degrees}
\subsection{Reference Angles}
\input{../../freecalc/modules/trigonometry/reference-angles}
\input{../../freecalc/modules/trigonometry/trig-example}
\input{../../freecalc/modules/trigonometry/frequently-encountered-sines-cosines}
\subsection{Geometric Interpretation of the Trigonometric Functions}
\input{../../freecalc/modules/trigonometry/trig-functions-unit-circle}
\subsection{Periodicity and Symmetries of the Trig Functions}
\input{../../freecalc/modules/trigonometry/trig-functions-even-odd}
\input{../../freecalc/modules/trigonometry/trig-functions-periodicity}
\input{../../freecalc/modules/trigonometry/trig-identity-definition}
\input{../../freecalc/modules/trigonometry/trig-identities-pythagorean}
%\input{../../modules/trigonometry/trig-identities-angle-sum-summary}
}

\lect{\semester}{Lecture 3}{3}{% begin lecture
%DesiredLectureName: Trigonometry_Cofunction_Identities_Angle_Sum_Formulas
\section{Cofunction identities}
\input{../../modules/trigonometry/cofunction-identities-part1}
\input{../../modules/trigonometry/cofunction-identities-part2}
\input{../../modules/trigonometry/cofunction-identities-memorization-aid}
\section{Trigonometric Functions of Sums of Angles}
\input{../../modules/trigonometry/similar-triangles-def}
\input{../../modules/trigonometry/similar-triangles-same-side-ratios}
\input{../../modules/trigonometry/trig-angle-sum-formulas-geometric-proof}
\input{../../modules/trigonometry/trig-angle-sum-formulas-geometric-proof-part2}
\input{../../modules/trigonometry/trig-angle-sum-formulas-geometric-proof-part3}
\input{../../modules/trigonometry/trig-angle-sum-formulas-to-find-sin-cos-ex1}
\input{../../modules/trigonometry/trig-angle-sum-formulas-sums-pi-over-2-ex1}
\input{../../modules/trigonometry/trig-angle-sum-formulas-sums-pi-over-2-ex2}
\input{../../modules/trigonometry/trig-angle-sum-formulas-tan-is-pi-periodic}
\input{../../modules/trigonometry/trig-angle-sum-formulas-to-prove-pythagorean-identity}
\input{../../modules/trigonometry/tan-angle-sum-formula}
\section{Double Angle Formulas}
\input{../../modules/trigonometry/double-angle-formulas}
\input{../../modules/trigonometry/double-angle-formulas-proof}
\input{../../modules/trigonometry/half-angle-formula-ex1}
\input{../../modules/trigonometry/power-reducing-formulas}
\input{../../modules/trigonometry/power-reducing-formulas-ex1}
}

\lect{\semester}{Lecture 4}{4}{% begin lecture
%DesiredLectureName: Complex-numbers-basics
\section{Complex Numbers}
\input{../../modules/complex-numbers/complex-numbers-definition}
\input{../../modules/complex-numbers/complex-numbers-addition-multiplication-example-1}
\input{../../modules/complex-numbers/complex-numbers-multiplication-example-2}
\input{../../modules/complex-numbers/overview-of-numbers}
}

\lect{\semester}{Lecture 5}{5}{% begin lecture
%DesiredLectureName: Trigonometric_Identities
\section{Trigonometric Identities}
\subsection{Trigonometric Identities and Complex Numbers}
\input{../../modules/trigonometry/trig-Euler-formula}
\input{../../modules/trigonometry/trig-using-euler-formula}
\subsection{Trigonometric Identities without Complex Numbers}
\input{../../modules/trigonometry/trig-identity-definition}
\input{../../modules/trigonometry/trig-identity-proving}
\input{../../modules/trigonometry/trig-identity-types}
\subsection{Trig Identities Using $\sin^2\theta+\cos^2\theta=1$}
\input{../../modules/trigonometry/trig-identities-pythagorean-ex1}
\input{../../modules/trigonometry/trig-identities-pythagorean-ex2}
\input{../../modules/trigonometry/trig-identities-pythagorean-ex3}
\input{../../modules/trigonometry/trig-identities-pythagorean-ex4}
\subsection{Trig Identities Using the Angle Sum Formulas}
\input{../../modules/trigonometry/trig-identities-example-1}
\input{../../modules/trigonometry/trig-identities-example-3}
\input{../../modules/trigonometry/trig-identities-example-3-with-Eulers-Formula}
\input{../../modules/trigonometry/trig-identities-example-2}
\input{../../modules/trigonometry/trig-identities-strategy-note}
\subsection{Trig Identities Exercises}
\input{../../modules/trigonometry/trig-identities-exercises-1}
}

\lect{\semester}{Lecture 6}{6}{% begin lecture
%DesiredLectureName: Inverse_Functions
\section{Inverse Functions}
\subsection{One-to-one Functions}
\input{../../modules/inverse-functions/one-to-one-def}
\input{../../modules/inverse-functions/horizontal-line-test}
\subsection{The Definition of the Inverse of $f$}
\input{../../modules/inverse-functions/inverse-function-def}
\input{../../modules/inverse-functions/inverse-notation-warning}
\input{../../modules/inverse-functions/inverse-function-equations}
\input{../../modules/inverse-functions/inverse-function-solve-for-version2}
\input{../../modules/inverse-functions/guess-and-check}
\input{../../modules/inverse-functions/inverse-function-graph}
\input{../../modules/inverse-functions/inverse-function-ex5}
\input{../../modules/inverse-functions/inverse-function-solve-for-ex1-freeCalc}
\input{../../modules/inverse-functions/inverse-function-solve-for-ex3-freeCalc}
}

\lect{\semester}{Lecture 7}{7}{% begin lecture
%DesiredLectureName: Trigonometric_Function_Graphs_Inverse_Trig
\section{Graphs of the Trigonometric Functions}
\subsection{Graphs of $\sin $ and $\cos$}
\input{../../freecalc/modules/trigonometry/graph-sin-extended}
\input{../../freecalc/modules/trigonometry/graph-cos-extended}
\input{../../freecalc/modules/trigonometry/graphs-sin-and-cos}
\subsection{Graph of $a\sin(bx-c)$}
\input{../../freecalc/modules/trigonometry/graphs-asinbx-plus-c}
\subsection{Graphs of $\tan, \cot, \sec, \csc $}
\input{../../freecalc/modules/trigonometry/graph-tan-extended}
\input{../../freecalc/modules/trigonometry/graphs-tan-and-cot}
\input{../../freecalc/modules/trigonometry/graphs-sec-and-csc}
\section{Inverse Trigonometric Functions}
\input{../../freecalc/modules/inverse-trig/arcsin-def}
\input{../../freecalc/modules/inverse-trig/arcsin-frequently-encountered-example-1}
\input{../../freecalc/modules/inverse-trig/tan-arcsin-example-1}
\input{../../freecalc/modules/inverse-trig/arcsin-sin-ex1}
\input{../../freecalc/modules/inverse-trig/arcsin-sin-ex2}
\input{../../freecalc/modules/inverse-trig/arcsin-properties}
\input{../../freecalc/modules/inverse-trig/arccos-def}
\input{../../freecalc/modules/inverse-trig/arccos-properties}
\input{../../freecalc/modules/inverse-trig/arccos-cos-example-1}
%\section{Some applied trigonometric problems}
\input{../../freecalc/modules/trigonometry/text-problem-visibility-horizon-earth-1}
%\input{../../freecalc/modules/trigonometry/text-problem-visibility-mountain-earth-1}
\subsection{Trigonometric Functions with Inverse Trig Arguments}
\input{../../freecalc/modules/inverse-trig/trig-applied-to-inverse-trig-ex1}
\input{../../freecalc/modules/inverse-trig/trig-applied-to-inverse-trig-ex2}
\input{../../freecalc/modules/inverse-trig/arctan-def}
\input{../../freecalc/modules/inverse-trig/arctan-ex3}
\input{../../freecalc/modules/inverse-trig/arcsec-intro}
\input{../../freecalc/modules/inverse-trig/arcsec-def}
}

\lect{\semester}{Lecture 8}{8}{% begin lecture
%DesiredLectureName: Trig_Equations_Inequalities
\section{Trigonometric equations and inequalities}
\input{../../freecalc/modules/trigonometry/trig-equations-intro}
\subsection{The Equations $\sin x = A$, $\cos x = B$}
\input{../../freecalc/modules/trigonometry/trig-equations-sin-x-equals-const-algebraic-1}
\input{../../freecalc/modules/trigonometry/trig-equations-cos-x-equals-const-algebraic-1}
\input{../../freecalc/modules/trigonometry/trig-equations-sin-x-equals-const-general-1}
\subsection{Equations that reduce to $\sin x = A$, $\cos x = B$}
\input{../../freecalc/modules/trigonometry/trig-equations-example}
\input{../../freecalc/modules/trigonometry/trig-equations-example-2}
\input{../../freecalc/modules/trigonometry/trig-equations-strategy}
\section{Product-to-Sum Formulas}
\input{../../freecalc/modules/trigonometry/product-to-sum-formulas}
\input{../../freecalc/modules/trigonometry/sum-to-product-formulas}
\input{../../freecalc/modules/trigonometry/sum-to-product-formulas-ex1}
\section{Trigonometric inequalities}
\input{../../freecalc/modules/trigonometry/sin-geq-leq-constants-example-1}
\input{../../freecalc/modules/inequalities-polynomial/trig-inequality-reduced-to-quadratic-1}
}

\lect{\semester}{Lecture 9}{9}{% begin lecture
%DesiredLectureName: Law_of_Sines_Law_of_Cosines_Trig_Applications
\section{Law of sines}
\input{../../modules/trigonometry/area-of-triangle-from-base-and-height}
\input{../../modules/trigonometry/area-of-triangle-from-two-sides-and-angle}
\input{../../modules/trigonometry/law-of-sines}
\input{../../modules/trigonometry/law-of-sines-example-1}
\section{Law of cosines}
\input{../../modules/trigonometry/law-of-cosines}
\input{../../modules/trigonometry/law-of-cosines-solve-triangle-example-1}
}

\lect{\semester}{Lecture 10}{10}{
%DesiredLectureName: Exponents
\section{Exponents}
\input{../../modules/exponential-functions/exponential-properties}
\subsection{Two ways to define exponents}
\input{../../modules/exponential-functions/exponential-function-def-various-approaches}
\input{../../modules/exponential-functions/exponential-function-def}
\subsection{Basic properties}
\input{../../modules/exponential-functions/exponential-function-graphs}
\input{../../modules/exponential-functions/exponential-versus-polynomial}
\input{../../modules/exponential-functions/exponential-function-ex-sketch}
\input{../../modules/exponential-functions/exponential-one-to-one-over-reals}
\subsection{The Natural Exponential Function}
\input{../../modules/exponential-functions/natural-exponential-intro}
\input{../../modules/logarithms/e-limit-no-lim-notation}
\input{../../modules/exponential-functions/compound-interest}
\input{../../modules/logarithms/e-limit-compound-interest-example-1}
\input{../../modules/logarithms/e-limit-compound-interest-example-2}
\input{../../modules/logarithms/e-limit-compound-interest-rule-of-72}
}

\lect{\semester}{Lecture 11}{11}{% begin lecture
%DesiredLectureName: Logarithm_Basics
\section{Logarithmic Functions}
\subsection{Logarithm basics}
\input{../../modules/logarithms/logarithm-def}
\input{../../modules/logarithms/logarithm-def-ex1}
\input{../../modules/logarithms/log-and-exp}
\input{../../modules/logarithms/logarithm-graphs}
\subsection{Natural Logarithms}
\input{../../modules/logarithms/natural-logarithm-def}
\input{../../modules/logarithms/log-notation-note}
\input{../../modules/logarithms/log-notation-summary}
\subsection{Shifting graphs of logarithmic functions}
\input{../../modules/logarithms/natural-logarithm-def-ex8}
\section{Basic Operations with Logarithms}
\input{../../modules/logarithms/logarithm-properties}
\input{../../modules/logarithms/logarithm-properties-compute-arbitrary-base-via-ln-example-1}
\input{../../modules/logarithms/logarithm-properties-ex2}
\input{../../modules/logarithms/logarithm-properties-ex4}
\input{../../modules/logarithms/logarithm-properties-ex6}
\input{../../modules/logarithms/logarithm-properties-ex7}
\input{../../modules/logarithms/logarithm-properties-extra}
\input{../../modules/logarithms/logarithm-properties-ex5}
\input{../../modules/logarithms/logarithm-evaluate-log-linear-combination-multiple-techniques-1}
\input{../../modules/logarithms/logarithm-properties-proofs}
}

\lect{\semester}{Lecture 12}{12}{% begin lecture
%DesiredLectureName: Equations_involving_logarithms_and_exponents
\section{Equations involving logarithms}
\input{../../modules/logarithms/logarithms-equations-ex1}
\input{../../modules/logarithms/logarithms-equations-ex2}
\section{Equations involving exponents}
\input{../../modules/exponential-functions/exponential-equation1}
\input{../../modules/logarithms/exponential-equation-basic-arbitrary-base-1.tex}
\input{../../modules/logarithms/exponential-equation-basic-two-exponents-1}
\input{../../modules/logarithms/exponential-equation-basic-two-different-exponents-1}
\input{../../modules/logarithms/natural-logarithm-def-ex5}
\input{../../modules/exponential-functions/exponential-word-problem1}
\input{../../modules/exponential-functions/exponential-equation2}
\input{../../modules/logarithms/exponential-equation-quadratic-natural-base-1}
\input{../../modules/logarithms/exponential-equation-quadratic-arbitrary-base-1}
\input{../../modules/logarithms/exponential-equation-quadratic-arbitrary-base-2}
\section{Inverse function problems and exponents}
\input{../../modules/inverse-functions/inverse-function-solve-for-ex2-freeCalc}
\section{Basic exponential inequalities}
\input{../../modules/logarithms/exponential-inequality-basic-1}
}

\lect{\semester}{Lecture 13}{13}{% begin lecture
%DesiredLectureName: Exponential_and_Logarithmic_models
\section{Models Involving Logarithms and Exponents}
\input{../../modules/logarithms/models-involving-logarithms-and-exponents}
\input{../../modules/exponential-growth-and-decay/exponential-growth-decay-model}
\input{../../modules/exponential-growth-and-decay/exponential-growth-model-ex1}
\input{../../modules/exponential-growth-and-decay/exponential-decay-model-ex1}
\input{../../modules/exponential-growth-and-decay/logistic-growth-model}
\input{../../modules/exponential-growth-and-decay/logistic-growth-model-ex1}
\input{../../modules/logarithms/logarithmic-models}
\input{../../modules/logarithms/logarithmic-models-ex1}
}

\lect{\semester}{Lecture 14}{14}{
%DesiredLectureName: Factoring_polynomials_and_polynomial_inequalities
\section{Factoring quadratics}
\input{../../modules/polynomial-factorization/factorization-meaning-1}
\input{../../modules/quadratic-functions/factoring-quadratics-theory-1}
\input{../../modules/quadratic-functions/factoring-quadratics-example-Vieta-guessing-or-formula-1}
\input{../../modules/quadratic-functions/vietas-formulas}
\input{../../modules/quadratic-functions/factoring-quadratics-example-example-Vieta-guessing-1}
\input{../../modules/quadratic-functions/factoring-quadratics-example-by-formula-1}
\input{../../modules/quadratic-functions/factoring-quadratics-example-by-formula-no-real-roots-1}
\section{Factorization overview}
\input{../../modules/polynomial-factorization/factorization-examples-no-theory-1}
\input{../../modules/polynomial-factorization/fundamental-theorem-of-algebra-statement}
\input{../../modules/polynomial-factorization/factorization-examples-1}
\input{../../modules/polynomial-factorization/factorization-algebraic-overview-1}
\input{../../modules/polynomial-factorization/factorization-over-basic-three-fields-explanation}
\input{../../modules/polynomial-factorization/factorization-over-rationals}
%DesiredLectureName: Factoring_cubics_with_rational_root
\section{Polynomial division}
\input{../../modules/polynomial-factorization/polynomial-division-example-1}
\input{../../modules/polynomial-factorization/polynomial-division-use-to-factor-1}
\section{Factoring cubics with rational root}
\input{../../modules/polynomial-factorization/solve-cubic-rational-roots-using-graph-1}
\input{../../modules/polynomial-factorization/solve-cubic-one-rational-two-real-roots-from-graph-1}
\input{../../modules/polynomial-factorization/solve-cubic-one-rational-two-complex-roots-from-graph-1}
\section{Polynomial inequalities}
%DesiredLectureName: Polynomial_inequalities
\input{../../freecalc/modules/inequalities-polynomial/inequality-quadratic-1}
\input{../../freecalc/inequalities-polynomial/inequality-cubic-three-rational-roots-1}
}

\lect{\semester}{Lecture 98}{98}{
%DesiredLectureName: Graphing-Equations-Circle-Equation
\section{Graph of an equation}
\input{../../modules/equation-graph/equation-graph-def}
\input{../../modules/equation-graph/equation-check-solution-1}
\input{../../modules/equation-graph/equation-check-solution-2}
\input{../../modules/equation-graph/graph-equation-when-one-variable-is-function-of-the-other}
\youWillNotBeTested
\input{../../modules/equation-graph/graph-equation-implicit-computer-algorithm}
\input{../../modules/equation-graph/graph-intercepts}
\input{../../modules/equation-graph/find-graph-intercepts-example-1}
\input{../../modules/equation-graph/find-graph-intercepts-example-2}
\input{../../modules/equation-graph/graph-symmetries}
\input{../../modules/equation-graph/circle-equation}
\input{../../modules/quadratic-functions/completing-the-square}
\input{../../modules/quadratic-functions/completing-the-square-example-1}
\input{../../modules/equation-graph/find-center-radius-from-circle-equation-1}
\input{../../modules/equation-graph/find-circle-equation-from-radius-and-point-1}
}% end lecture

\lect{\semester}{Lecture 99}{99}{
%DesiredLectureName: Quadratic_Functions
\section{Quadratic Functions}
\subsection{Standard Form}
\input{../../freecalc/modules/quadratic-functions/definition-quadratic-function}
\input{../../freecalc/modules/quadratic-functions/completing-the-square-example-1}
\input{../../freecalc/modules/quadratic-functions/completing-the-square}
\input{../../freecalc/modules/quadratic-functions/discriminant-and-standard-form-quadratic}
\subsection{Geometric Features}
\input{../../freecalc/modules/quadratic-functions/quadratic-function-geometric-features}
\input{../../freecalc/modules/quadratic-functions/write-equation-parabola-given-vertex-and-point-example-1}
\subsection{Quadratic Equations}
\input{../../freecalc/modules/quadratic-functions/quadratic-equation-solution}
\input{../../freecalc/modules/quadratic-functions/quadratic-equation-solution-example-1}
\input{../../freecalc/modules/quadratic-functions/quadratic-equation-solution-example-2}
\input{../../freecalc/modules/quadratic-functions/quadratic-equation-solution-example-3}
\subsection{Vieta's Formulas}
\input{../../freecalc/modules/quadratic-functions/dicriminant-from-roots-formula}
\input{../../freecalc/modules/quadratic-functions/vietas-formulas}
\subsection{Factoring quadratics}
\input{../../freecalc/modules/quadratic-functions/factoring-quadratics-theory-1}
\input{../../freecalc/modules/quadratic-functions/factoring-quadratics-example-Vieta-guessing-or-formula-1}
\input{../../freecalc/modules/quadratic-functions/vietas-formulas}
\input{../../freecalc/modules/quadratic-functions/factoring-quadratics-example-example-Vieta-guessing-1}
\input{../../freecalc/modules/quadratic-functions/factoring-quadratics-example-by-formula-1}
\input{../../freecalc/modules/quadratic-functions/factoring-quadratics-example-by-formula-no-real-roots-1}
\subsection{Plotting Quadratics}
\input{../../freecalc/modules/quadratic-functions/plot-roughly-by-hand-parabola-recipe}
\input{../../freecalc/modules/quadratic-functions/plot-roughly-by-hand-parabola-example-1}
\input{../../freecalc/modules/quadratic-functions/quadratic-find-parameters-to-ensure-quadratic-positive-1}
\subsection{Maxima and Minima}
\input{../../freecalc/modules/quadratic-functions/quadratic-function-minimum-example-1}
\input{../../freecalc/modules/quadratic-functions/quadratic-function-maximum-or-minimum}
\input{../../freecalc/modules/quadratic-functions/quadratic-max-min-word-problem-1}
}

\lect{\semester}{Lecture 100}{100}{
%DesiredLectureName: Cartesian_Coordinates_Vector_Addition_Scalar_Product
\section{Cartesian coordinate system}
\input{../../modules/coordinate-systems/cartesian-coordinates-2d-part1}
\input{../../modules/coordinate-systems/cartesian-coordinates-2d-part2}
\input{../../modules/coordinate-systems/cartesian-coordinates-2d-part3}
\input{../../modules/coordinate-systems/cartesian-coordinates-2d-plot-points-example}

\subsection{The Pythagorean Theorem, Euclidean Distance}
\input{../../modules/coordinate-systems/pythagorean-theorem}
\input{../../modules/coordinate-systems/distance-in-cartesian-coordinates-2d}
\input{../../modules/coordinate-systems/distance-2d-example-1}
\input{../../modules/coordinate-systems/distance-2d-check-if-triangle-right-angled-1}
\subsection{Vectors}
\input{../../modules/vectors/vector-addition-and-scalar-mult-from-cartesian-system-2d-part1}
\input{../../modules/vectors/vector-addition-and-scalar-mult-from-cartesian-system-2d-part2}
\input{../../modules/vectors/vector-addition-and-scalar-mult-from-cartesian-system-2d-part3}
\input{../../modules/vectors/vector-addition-and-scalar-mult-from-cartesian-system-2d-part4-translation-definition}
\input{../../modules/vectors/vectors-translation-of-point-2d-example-1}
\subsection{Segments, Midpoints}
\input{../../modules/coordinate-systems/segment-between-two-points}
\input{../../modules/coordinate-systems/midpoint-definition-verification-2d}
\input{../../modules/coordinate-systems/find-midpoint-1}
}% end lecture

\lect{\semester}{Lecture 101}{101}{
%DesiredLectureName: Lines
\section{Lines}
\input{../../modules/coordinate-systems/r-two-r-n}
\input{../../modules/lines-2d/line-definition}
\input{../../modules/lines-2d/line-from-two-points}
\input{../../modules/lines-2d/line-from-two-points-formulas-and-slope-form}
\subsection{Slope-intercept Form}
\input{../../modules/lines-2d/line-slope-intercept-form}
\input{../../modules/lines-2d/line-slope-intercept-form-explanation}
\input{../../modules/lines-2d/compare-linear-functions-via-slope}

\input{../../modules/lines-2d/vertical-line-and-slope}
\input{../../modules/lines-2d/line-plot-from-equation-example-1}

\input{../../modules/lines-2d/line-from-two-points-example-1}
\input{../../modules/lines-2d/line-from-point-and-slope-example-1}

\subsection{Line intersection}
\input{../../modules/lines-2d/line-intersection}
\input{../../modules/lines-2d/line-intersection-example-1}
\input{../../modules/lines-2d/parallel-lines-equal-slopes}
}% end lecture

\lect{\semester}{Lecture 102}{102}{
%DesiredLectureName: Functions
\section{The Definition of a Function}
\input{../../modules/functions-basics/function-def-uses-arrows}
\input{../../modules/functions-basics/function-def}
\input{../../modules/functions-basics/function-def-note-on-f-of-x-notation}
\input{../../modules/functions-basics/function-formulas-and-bound-variables}
\input{../../modules/functions-basics/function-formula-understanding-example-1}
\subsection{Function Domains}
\input{../../modules/functions-basics/domains}
\input{../../modules/functions-basics/domains-example-1}

\subsection{The Vertical Line Test}
\input{../../modules/precalculus/vertical-line-test}
\subsection{Piecewise Defined Functions}
\input{../../modules/precalculus/function-piecewise}
\input{../../modules/precalculus/absolute-value}
\input{../../modules/precalculus/piecewise-formula}
\input{../../modules/precalculus/piecewise-ex1}
\input{../../modules/precalculus/piecewise-ex2}
\subsection{Zeros of a function}
\input{../../modules/functions-basics/zeroes-of-a-function}
\input{../../modules/functions-basics/zeroes-of-function-from-formula-1}
\input{../../modules/functions-basics/f-of-x-equals-g-of-x-algebra-example-1}
\input{../../modules/functions-basics/finding-when-f-of-x-equals-g-of-x}
\subsection{Symmetry}
\input{../../modules/precalculus/even-and-odd}
\subsection{Increasing and Decreasing Functions}
\input{../../modules/precalculus/increasing-decreasing}

}% end lecture

\lect{\semester}{Lecture 103}{103}{
%DesiredLectureName: Functions
\section{A Catalog of Essential Functions}
\subsection{Linear Functions}
\input{../../modules/lines-2d/linear-functions}
\input{../../modules/lines-2d/linear-functions-examples-1}
\subsection{Polynomials}
\input{../../modules/precalculus/polynomials}
\subsection{Power Functions}
\input{../../modules/precalculus/power-functions-def}
\input{../../modules/precalculus/root-functions}
\input{../../modules/precalculus/reciprocal-function}
\subsection{Rational Functions}
\input{../../modules/precalculus/rational-functions}
\subsection{Algebraic Functions}
\input{../../modules/precalculus/algebraic-functions}
\subsection{Transcendental Functions}
\input{../../modules/precalculus/transcendental-functions-version2}
\subsection{Miscellaneous}
\input{../../modules/continuity/greatest-integer-function}
\section{New Functions from Old Functions}
\input{../../modules/precalculus/combinations-functions}
\input{../../modules/precalculus/composition-functions}
\input{../../modules/precalculus/composition-example}
\input{../../modules/precalculus/composition-example-2}
}

\lect{\semester}{Lecture 104}{104}{
\section{Composing Functions with Linear Transformations}
\input{../../modules/function-graph-linear-transformations/transformations-shifts}
\input{../../modules/function-graph-linear-transformations/transformations-shifts-example}
\input{../../modules/function-graph-linear-transformations/transformations-magnifications}
\input{../../modules/function-graph-linear-transformations/transformations-horizontal-stretches}
\section{Graphing Absolute Value of a Function}
\input{../../modules/function-graph-linear-transformations/transformations-absolute-value}
}
\end{document}
