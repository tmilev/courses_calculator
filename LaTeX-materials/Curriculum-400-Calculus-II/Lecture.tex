\documentclass
%[handout]
{beamer}
\newcommand{\semester}{2018}
\usepackage[breakwords]{truncate}
\input{../../freecalc/lectures/example-templates}
\input{../../freecalc/lectures/system-specific-config-Ubuntu-texlive} %
\input{../../freecalc/lectures/pstricks-commands}
\renewcommand{\Arcsin}{\arcsin}
\renewcommand{\Arccos}{\arccos}
\renewcommand{\Arctan}{\arctan}
\renewcommand{\Arccot}{\text{arccot\hspace{0.03cm}}}
\renewcommand{\Arcsec}{\text{arcsec\hspace{0.03cm}}}
\renewcommand{\Arccsc}{\text{arccsc\hspace{0.03cm}}}
\useinnertheme{rounded}
\useoutertheme{infolines}
\usecolortheme{orchid}
\usecolortheme{whale}


\usepackage[T1]{fontenc}
% Or whatever. Note that the encoding and the font should match. If T1
% does not look nice, try deleting the line with the fontenc.

\setbeamertemplate{navigation symbols}{}

\newcommand{\lect}[4]{
\ifnum#3=\currentLecture
  \date{#1}
  \lecture[#1]{#2}{#3}
#4
\else
%include nothing
\fi
}

\setbeamertemplate{headline}
{
  \leavevmode%
  \hbox{%
  \begin{beamercolorbox}[wd=.4\paperwidth,ht=2.25ex,dp=1ex,center]{author in head/foot}%
    \usebeamerfont{title in head/foot}{\insertsectionhead}
  \end{beamercolorbox}%
  \begin{beamercolorbox}[wd=.4\paperwidth,ht=2.25ex,dp=1ex,center]{title in head/foot}%
    \usebeamerfont{title in head/foot}{\insertsubsectionhead}
  \end{beamercolorbox}%
\begin{beamercolorbox}[wd=.2\paperwidth,ht=2.25ex,dp=1ex,center]{date in head/foot}%
\insertframenumber/\inserttotalframenumber
  \end{beamercolorbox}
}
}

\setbeamertemplate{footline}
{
  \leavevmode%
  \hbox{%
  \begin{beamercolorbox}[wd=.4\paperwidth,ht=2.25ex,dp=1ex,center]{author in head/foot}%
    \usebeamerfont{author in head/foot}\insertshortauthor
  \end{beamercolorbox}%
  \begin{beamercolorbox}[wd=.4\paperwidth,ht=2.25ex,dp=1ex,center]{title in head/foot}%
    \usebeamerfont{title in head/foot}{\insertshorttitle}
  \end{beamercolorbox}%
  \begin{beamercolorbox}[wd=.2\paperwidth,ht=2.25ex,dp=1ex,center]{date in head/foot}%
    \usebeamerfont{date in head/foot}\insertshortdate{}
  \end{beamercolorbox}}%
  \vskip0pt%
}


\AtBeginLecture{%
\title[{\truncate{4.2cm}{\insertlecture}}]{
\textbf{\color{green}{Cryptography 101}}
\\
\textbf{\insertlecture} \\
{\color{green}{\textbf{calculator-algebra.org}}}
}
\author[{\color{green}{\bf calculator-algebra.org}}, Todor Milev]{
~\\~\\~\\
\begin{tabular}{c}
\Large Todor Milev 
\end{tabular}
}


\date{\insertshortlecture}

\begin{frame}
\titlepage
\end{frame}
}

\begin{document}
\providecommand{\currentLecture}{1}


\lect{\semester}{Lecture (not covered in class)}{0}{
%DesiredLectureName: Exponentials_Review_Not_Covered_In_Class
\section{Exponential Functions and logarithms, Review}
\input{../../modules/exponential-functions/exponential-function-def}
\input{../../modules/exponential-functions/exponential-function-graphs}
\section{Derivatives of Exponential Functions}
\input{../../modules/exponential-functions/exponential-function-derivative}
\subsection{Natural Exponent}
\input{../../modules/exponential-functions/natural-exponential-intro}
%\input{../../modules/exponential-functions/e-def}
\input{../../modules/exponential-functions/natural-exponential-def}
\section{A More Advanced Approach to Exponents}
\input{../../modules/exponential-functions/exponential-function-def-various-approaches}
\subsection{Derivative using series definition}
\input{../../modules/derivatives-computation-understanding/exponent-derivative-from-taylor-series}
\section{Logarithmic Functions, Review}
\input{../../modules/logarithms/logarithm-def}
\input{../../modules/logarithms/logarithm-def-ex1}
\input{../../modules/logarithms/log-and-exp}
\input{../../modules/logarithms/logarithm-graphs}
\subsection{Natural Logarithms}
\input{../../modules/logarithms/natural-logarithm-def}
\input{../../modules/logarithms/logarithm-properties}
%\input{../../modules/logarithms/natural-logarithm-def-ex8}
%\input{../../modules/inverse-functions/inverse-function-solve-for-ex2-freeCalc}
\section{Derivatives of Logarithms, Review}
\subsection{The Natural Logarithm}
\input{../../modules/logarithms/natural-logarithm-derivative}
\input{../../modules/logarithms/natural-logarithm-derivative-ex1}
\subsection{The Number $e$ as a Limit}
\input{../../modules/logarithms/e-limit}
\input{../../modules/logarithms/e-limit-problems-ex1}
\subsection{Derivatives of Exponents with Arbitrary Base}
\input{../../modules/logarithms/general-exponential-derivative}
\input{../../modules/logarithms/general-exponential-derivative-ex13}
\subsection{Derivatives of Arbitrary Exponents with Arbitrary Base}
\input{../../modules/logarithms/arbitrary-base-arbitrary-exponent-derivative-ex1}
\input{../../modules/logarithms/arbitrary-base-arbitrary-exponent-derivative}
} %end lecture

% begin lecture
\lect{\semester}{Lecture  1}{1}{
%DesiredLectureName: Trigonometry_Review_Inverse_Trigonometry
\section{Review of trigonometry}
\subsection{The Trigonometric Functions}
\input{../../freecalc/modules/trigonometry/trig-functions-unit-circle}
\subsection{Trigonometric Identities}
\input{../../freecalc/modules/trigonometry/trig-identity-definition}
\input{../../freecalc/modules/trigonometry/trig-identities-pythagorean}
\input{../../freecalc/modules/trigonometry/trig-identities-angle-sum-summary}
\input{../../freecalc/modules/trigonometry/trig-functions-even-odd}
\input{../../freecalc/modules/trigonometry/trig-functions-periodicity}
\subsection{Trigonometric Identities and Complex Numbers}
\input{../../freecalc/modules/complex-numbers/complex-numbers-definition}
\input{../../freecalc/modules/trigonometry/trig-Euler-formula}
\input{../../freecalc/modules/trigonometry/trig-using-euler-formula}
\input{../../freecalc/modules/trigonometry/trig-identities-example-3-with-Eulers-Formula}
\subsection{Graphs of the Trigonometric Functions}
\input{../../freecalc/modules/trigonometry/graphs-sin-and-cos}
\input{../../freecalc/modules/trigonometry/graphs-tan-and-cot}
\input{../../freecalc/modules/trigonometry/graphs-sec-and-csc}
\section{Inverse Trigonometric Functions}
\input{../../freecalc/modules/inverse-trig/arcsin-def}
\input{../../freecalc/modules/inverse-trig/arcsin-frequently-encountered-example-1}
\input{../../freecalc/modules/inverse-trig/tan-arcsin-example-1}
\input{../../freecalc/modules/inverse-trig/arcsin-sin-ex1}
\input{../../freecalc/modules/inverse-trig/arcsin-sin-ex2}
\input{../../freecalc/modules/inverse-trig/arcsin-derivative}
\input{../../freecalc/modules/inverse-trig/arcsin-properties}
\input{../../freecalc/modules/inverse-trig/arccos-def}
\input{../../freecalc/modules/inverse-trig/arccos-properties}
\input{../../freecalc/modules/inverse-trig/trig-applied-to-inverse-trig-ex1}
\input{../../freecalc/modules/inverse-trig/trig-applied-to-inverse-trig-ex2}
\input{../../freecalc/modules/inverse-trig/arctan-def}
\input{../../freecalc/modules/inverse-trig/arctan-ex3}
\input{../../freecalc/modules/inverse-trig/arctan-ex4}
\input{../../freecalc/modules/inverse-trig/arctan-derivative}
\input{../../freecalc/modules/inverse-trig/arcsec-intro}
\input{../../freecalc/modules/inverse-trig/arcsec-def}
\input{../../freecalc/modules/inverse-trig/inverse-trig-derivatives-summary}
\input{../../freecalc/modules/inverse-trig/arcsin-ex5}
\input{../../freecalc/modules/inverse-trig/inverse-trig-reminder}
}% end lecture

%begin lecture
\lect{\semester}{Lecture 2}{2}{
%DesiredLectureName: Integration_Basics_Review
\section{Integration, Review}
\subsection{The Evaluation Theorem (FTC part 2)}
\input{../../modules/antiderivatives/antiderivative-def}
\input{../../modules/integration/FTC-part2}
\input{../../modules/integration/indefinite-integral-intro}
\section{Integration Techniques from Calc I, Review}
\subsection{Differential Forms, Review}
\input{../../modules/differentials/differential-def-version2}
\input{../../modules/differentials/differentials-integration-connection-intro}
\input{../../modules/differentials/differentials-integration-rules}
\section{Integration and Logarithms, Review}
\input{../../modules/logarithms/power-rule-minus-one}
} % end lecture

% begin lecture
\lect{\semester}{Lecture  3}{3}{
%DesiredLectureName: Integration_By_Parts
\section{Integration by Parts}
\input{../../modules/integration-by-parts/integration-by-parts-intro-version2}
\input{../../modules/integration-by-parts/integration-by-parts-LIPET-mnemonic}
\input{../../modules/integration-by-parts/integration-by-parts-ex1-version2}
\input{../../modules/integration-by-parts/integration-by-parts-ex2-version2}
\input{../../modules/integration-by-parts/integration-by-parts-ex3-version2}
\input{../../modules/integration-by-parts/integration-by-parts-ex4-version2}
\input{../../modules/integration-by-parts/integration-by-parts-ex5-version2}
}% end lecture

% begin lecture
\lect{\semester}{Lecture 4}{4}{
%DesiredLectureName: Integration_Rational_Functions_Bulding_Blocks
\section{Integration of Rational Functions}
\subsection{Building block integrals}
\input{../../modules/partial-fractions/partial-fractions-building-blocks-intro}
\input{../../modules/partial-fractions/partial-fractions-building-blocks-the-3-types}
\input{../../modules/partial-fractions/partial-fractions-building-block-1a}
\input{../../modules/partial-fractions/partial-fractions-building-block-1b}
\input{../../modules/partial-fractions/partial-fractions-building-blocks-2a-and-3a-intro}
\input{../../modules/partial-fractions/partial-fractions-building-blocks-2a-quadratic-no-lin-term}
\input{../../modules/partial-fractions/partial-fractions-building-blocks-3a-quadratic-no-lin-term}
\input{../../modules/partial-fractions/partial-fractions-building-blocks-2a-and-3a-quadratic-no-real-roots-intro}
\input{../../modules/partial-fractions/partial-fractions-building-blocks-2a-and-3a-quadratic-no-real-roots-ex1}
\input{../../modules/partial-fractions/partial-fractions-building-block-2b}

\input{../../modules/partial-fractions/partial-fractions-building-block-3b-example-n-equals-2}
\input{../../modules/partial-fractions/partial-fractions-building-block-3b}
\input{../../modules/partial-fractions/partial-fractions-building-blocks-linear-substitutions-summary}
}

\lect{\semester}{Lecture 5}{5}{
%DesiredLectureName: Integration_Rational_Functions_Partial_Fractions
\section{Integration of Rational Functions}
\subsection{Partial fractions}
\input{../../modules/partial-fractions/from-building-blocks-to-complete-algorithm-intro}
\input{../../modules/partial-fractions/partial-fractions-definition}
\input{../../modules/partial-fractions/partial-fractions-long-division}
\input{../../modules/partial-fractions/partial-fractions-long-division-ex1}
\input{../../modules/partial-fractions/rational-functions-to-partial-fractions}
\input{../../modules/partial-fractions/partial-fractions-case1}
\input{../../modules/partial-fractions/partial-fractions-case1-ex2}
\input{../../modules/partial-fractions/partial-fractions-case1-quick-trick}
\input{../../modules/partial-fractions/partial-fractions-case2}
\input{../../modules/partial-fractions/partial-fractions-case2-ex4}
\input{../../modules/partial-fractions/partial-fractions-case3}
\input{../../modules/partial-fractions/partial-fractions-case3-ex5}
\input{../../modules/partial-fractions/partial-fractions-case4}
\input{../../modules/partial-fractions/partial-fractions-case4-ex7}
}% end lecture

% begin lecture
\lect{\semester}{Lecture 6}{6}{
%DesiredLectureName: Trig_Integrals
\section{Trigonometric Integrals}
\subsection{Integrating rational trigonometric integrals}
\input{../../modules/trig-integrals/trig-integrals-rationalizing-substitution}
\input{../../modules/trig-integrals/trig-integrals-rationalizing-substitution-ex1}
\input{../../modules/trig-integrals/trig-integrals-rationalizing-substitution-integral-of-sec}
\subsection{Ad hoc methods for trigonometric integrals}
\input{../../modules/trig-integrals/trig-integrals-without-rationalizing-substitution}
\input{../../modules/trig-integrals/trig-integrals-ex9}
\input{../../modules/trig-integrals/trig-integrals-ex10}
\input{../../modules/trig-integrals/trig-integrals-sin-cos-version2}
\input{../../modules/trig-integrals/trig-integrals-ex11}
%\input{../../modules/trig-integrals/trig-integrals-ex5}
%\input{../../modules/trig-integrals/trig-integrals-ex6}
\input{../../modules/trig-integrals/trig-integrals-ex12}
\input{../../modules/trig-integrals/trig-integrals-ex13}
\input{../../modules/trig-integrals/trig-integrals-tann-secm-strategy}

\input{../../modules/trig-integrals/trig-integrals-tan-sec}
\input{../../modules/trig-integrals/trig-integrals-ex7}
%\input{../../modules/trig-integrals/trig-integrals-ex8}
\input{../../modules/trig-integrals/trig-integrals-ex8-version2}
\input{../../modules/trig-integrals/trig-integrals-sinm-cosn}
\input{../../modules/trig-integrals/trig-integrals-sinm-cosn-ex9}
}% end lecture

\lect{\semester}{Lecture 7}{7}{
%DesiredLectureName: Integration_Radicals_Of_Quadratics
\section[\truncate{5.2cm}{Integrals of form $\int R(x, \sqrt{ ax^2+bx+c}) \diff x$, $R$ - rational function} ]{Integrals of form $\int R(x,\sqrt{ax^2+bx+c}) \diff x$, $R$ - rational function}
\input{../../modules/trig-substitution/quadratic-radicals-integrals-intro}
\input{../../modules/trig-substitution/trig-substitutions-intro}
\subsection[\truncate{5.2cm}{Transforming to the forms $\sqrt{x^2+1}, \sqrt{-x^2+1}, \sqrt{x^2-1} $}]{Transforming to the forms $\sqrt{x^2+1}, \sqrt{-x^2+1}, \sqrt{x^2-1} $}
\input{../../modules/trig-substitution/quadratic-radicals-linear-substitution-preparation-ex1}
\input{../../modules/trig-substitution/quadratic-radicals-linear-substitution-preparation-ex2}
\subsection{Table of Euler and trig substitutions}
\input{../../modules/trig-substitution/trig-substitutions-euler-substitutions-table}
\subsection{The case $\sqrt{x^2+1}$}
\input{../../modules/trig-substitution/trig-substitution-case-1-cot}
\input{../../modules/trig-substitution/trig-substitutions-ex1-freecalc}
%\input{../../modules/trig-substitution/trig-substitutions-ex3}
\input{../../modules/trig-substitution/Euler-substitution-case-1-cot}
\input{../../modules/trig-substitution/integral-radical-x-squared-plus-one}
\input{../../modules/trig-substitution/area-under-hyperbola-ex1}
\input{../../modules/trig-substitution/trig-substitutions-ex4}
\subsection{The case $\sqrt{-x^2+1}$}
\input{../../modules/trig-substitution/trig-substitution-case-2-cos}
\input{../../modules/trig-substitution/trig-substitutions-ex1}
\input{../../modules/trig-substitution/trig-substitutions-ex2}
\input{../../modules/trig-substitution/trig-substitutions-ex7}
\input{../../modules/trig-substitution/Euler-substitution-case-2-cos}
\subsection{The case $\sqrt{x^2-1}$}
\input{../../modules/trig-substitution/trig-substitution-case-3-sec}
\input{../../modules/trig-substitution/trig-substitutions-ex5}
\input{../../modules/trig-substitution/trig-substitutions-ex1-Euler-sub}
\input{../../modules/trig-substitution/Euler-substitution-case-3-sec}
\section{Rationalizing Substitutions}
\input{../../modules/partial-fractions/partial-fractions-rationalize-intro}
\input{../../modules/partial-fractions/partial-fractions-rationalize-ex9}
} %end lecture

% begin lecture
\lect{\semester}{Lecture  8}{8}{
%DesiredLectureName: LHospitals_Rule
\section{Indeterminate Forms and L'Hospital's Rule}
\input{../../modules/lhospital/lhospital-intro}
\input{../../modules/lhospital/lhospital-statement}
\input{../../modules/lhospital/lhospital-ex1}
\input{../../modules/lhospital/lhospital-ex2}
\subsection{Indeterminate Products}
\input{../../modules/lhospital/indeterminate-products-def}
\input{../../modules/lhospital/indeterminate-products-ex6}
\subsection{Indeterminate Differences}
\input{../../modules/lhospital/indeterminate-differences-def}
\input{../../modules/lhospital/indeterminate-differences-ex8}
\subsection{Indeterminate Powers}
\input{../../modules/lhospital/indeterminate-powers-def}
\input{../../modules/lhospital/indeterminate-powers-ex10}
\input{../../modules/lhospital/e-power-k-limit-by-lhospital-rule}
}% end lecture

% begin lecture
\lect{\semester}{Lecture 9}{9}{
%DesiredLectureName: Improper_Integrals
\section{Improper Integrals}
\input{../../modules/improper-integrals/improper-integral-def}
\subsection{Type I: Infinite Intervals}
\input{../../modules/improper-integrals/improper-integral-geometry}
\input{../../modules/improper-integrals/improper-integral-type1}
\input{../../modules/improper-integrals/improper-integral-type1-ex1}
\input{../../modules/improper-integrals/improper-integral-type1-ex3}
\input{../../modules/improper-integrals/improper-integral-type1-ex4}
\subsection{Type II: Discontinuous Integrands}
\input{../../modules/improper-integrals/improper-integral-type2}
\input{../../modules/improper-integrals/improper-integral-type2-ex5}
\input{../../modules/improper-integrals/improper-integral-type2-ex7}
\subsection{A Comparison Test for Improper Integrals}
\input{../../modules/improper-integrals/improper-integral-comparison}
\input{../../modules/improper-integrals/improper-integral-comparison-ex9}
\input{../../modules/improper-integrals/improper-integral-comparison-ex10}
\input{../../modules/improper-integrals/improper-integral-type1-arctan-geometric-interpretation}
}% end lecture

% begin lecture
\lect{\semester}{Lecture 10}{10}{
%DesiredLectureName: Polar_Coordinates
\section{Polar Coordinates}
\input{../../modules/polar-coordinates/polar-intro}
\input{../../modules/polar-coordinates/polar-questions}
\input{../../modules/polar-coordinates/polar-many-representations}
\input{../../modules/polar-coordinates/polar-two-points-coincide-iff}
\input{../../modules/polar-coordinates/polar-to-cartesian}
\input{../../modules/polar-coordinates/polar-to-cartesian-ex2}
\input{../../modules/polar-coordinates/polar-to-cartesian-ex3}
} %end lecture

% begin lecture
\lect{\semester}{Lecture 11}{11}{
%DesiredLectureName: Curves_Polar_Curves_Basics
\section{Curves}
\input{../../modules/parametric-curves/parametric-intro}
\input{../../modules/parametric-curves/parametric-curve-definition}
\input{../../modules/parametric-curves/curve-image-definition-intro}
\input{../../modules/parametric-curves/curve-image-definition}
\input{../../modules/parametric-curves/parametric-curve-vs-curve-image-terminology}
\input{../../modules/parametric-curves/graphs-of-functions-as-curves}
\input{../../modules/parametric-curves/eliminate-parameter-ex}
\input{../../modules/parametric-curves/implicit-vs-explicit-parametrization}
\input{../../modules/parametric-curves/parametric-ex2}
\input{../../modules/parametric-curves/parametric-ex4}
\subsection{The Cycloid}
\input{../../modules/parametric-curves/cycloid-def}
\input{../../modules/parametric-curves/cycloid-equations-ex7}
\subsection{Polar Curves}
\input{../../modules/polar-curves/polar-curve-definition}
\input{../../modules/polar-curves/polar-curve-ex4}
\input{../../modules/polar-curves/polar-curve-ex6}
\input{../../modules/polar-curves/cardioid-ex7}
\input{../../modules/polar-curves/polar-curve-ex8}
\input{../../modules/polar-curves/polar-symmetry}
}% end lecture

% begin lecture
\lect{\semester}{Lecture 12}{12}{
%DesiredLectureName: Tangents_To_Curves_Arc_Length
\section{Tangents to Curves}
\input{../../modules/parametric-curves/parametric-tangents-definition}
\input{../../modules/parametric-curves/parametric-tangents-ex0}
\input{../../modules/parametric-curves/parametric-tangents-to-function-graphs}
\input{../../modules/parametric-curves/parametric-tangents-ex1}
\input{../../modules/parametric-curves/cycloid-tangents-ex2}
\subsection{Tangents to Polar Curves}
\input{../../modules/polar-curves/polar-tangents}
\input{../../modules/polar-curves/cardioid-tangents-ex9}
\section{Arc Length}
\input{../../modules/arc-length/arc-length-intro}
\input{../../modules/arc-length/arc-length-derivation-parametric}
\input{../../modules/arc-length/arc-length-def-parametric}
\input{../../modules/arc-length/arc-length-does-not-depend-on-parametrization-when-one-to-one}
\input{../../modules/arc-length/arc-length-function-graph-from-parametric-curve-length}
\input{../../modules/arc-length/arc-length-def}
\input{../../modules/arc-length/arc-length-ex1}
\input{../../modules/arc-length/arc-length-parabola-length}
\input{../../modules/arc-length/arc-length-parametric-curve-ex1}
% WARNING: This section needs an example in which y' = (a - b)^2 with 2ab = .5
\input{../../modules/arc-length/arc-length-half-ex}
\input{../../modules/parametric-curves/cycloid-arc-length-ex5}
\subsection{Arc Length in Polar Coordinates}
\input{../../modules/polar-curves/polar-arc-length-formula}
\input{../../modules/polar-curves/cardioid-arc-length-ex4}
}

%end lecture
\lect{\semester}{Lecture 13}{13}{
%DesiredLectureName: Area_Locked_By_Curve
\section{Areas Locked by Curves}
\input{../../modules/parametric-curves/parametric-area-formula}
\input{../../modules/parametric-curves/cycloid-area-ex3}
\section{Areas in Polar Coordinates}
\input{../../modules/polar-curves/polar-area-intro}
\input{../../modules/polar-curves/polar-area-justification}
\input{../../modules/polar-curves/polar-area-ex1}
\input{../../modules/polar-curves/polar-intersection-ex3}
\input{../../modules/polar-curves/polar-area-ex2}
}% end lecture

\lect{\semester}{Lecture 14}{14}{
%DesiredLectureName: TOPIC_SKIPPED_Surface_Area_Solid_of_Revolution
\section{Surface area of solid of revolution}
\input{../../modules/surface-area/surface-area-surface-of-revolution-about-x-axis-motivation}
\input{../../modules/surface-area/surface-area-surface-of-revolution-about-x-axis}
\input{../../modules/surface-area/surface-area-surface-of-revolution-about-x-axis-example-1}
\input{../../modules/surface-area/surface-area-surface-of-revolution-about-x-axis-example-2}
\input{../../modules/surface-area/surface-area-surface-of-revolution-about-y-axis-example-1}
}

% begin lecture
\lect{\semester}{Lecture 15}{15}{
%DesiredLectureName: Sequences
\section{Sequences}
\input{../../modules/sequences/sequence-intro}
\input{../../modules/sequences/notation}
\input{../../modules/sequences/notation-examples}
\input{../../modules/sequences/sequence-def-version2}
\input{../../modules/sequences/ways-to-define-sequences}
\input{../../modules/sequences/sequence-ex1}
\input{../../modules/sequences/find-terms}
\input{../../modules/sequences/sequence-find-formula-ex1}
\input{../../modules/sequences/sequence-find-formula-ex2}
\input{../../modules/sequences/sequence-find-formula-warning}
\input{../../modules/sequences/sequence-ex2}
\input{../../modules/sequences/arithmetic-def}
\input{../../modules/sequences/arithmetic-ex}
\input{../../modules/sequences/geometric-def}
\input{../../modules/sequences/geometric-ex}
\input{../../modules/sequences/sequence-plotting}
\input{../../modules/sequences/sequence-limit-def}
\input{../../modules/sequences/sequence-limit-function}
\input{../../modules/sequences/sequence-ex4}
\input{../../modules/sequences/sequence-limit-infinite}
\input{../../modules/sequences/sequence-limit-laws}
\input{../../modules/sequences/sequence-ex5}
\input{../../modules/sequences/sequence-ex6}
\input{../../modules/sequences/sequence-squeeze-theorem}
\input{../../modules/sequences/sequence-ex7}
\input{../../modules/sequences/sequence-function-composition}
\input{../../modules/sequences/sequence-function-composition-ex8}
\input{../../modules/sequences/sequence-ex9}
\input{../../modules/sequences/sequence-geometric-ex10}
\input{../../modules/sequences/sequence-geometric-theorem}
\input{../../modules/sequences/sequence-monotonic-def}
\input{../../modules/sequences/sequence-monotonic-ex}
\input{../../modules/sequences/sequence-bounded-def}
\input{../../modules/sequences/monotonic-sequence-theorem}
% WARNING: You need an example of the monotonic sequence theorem in use.
% Try using alternating convergents in the continued fraction expansion
% of some irrational number (the golden mean will work well).
}% end lecture

% begin lecture
\lect{\semester}{Lecture 16}{16}{
%DesiredLectureName: Series
\section{Series}
\input{../../modules/series/formal-series-def}
\input{../../modules/series/notation-through-example-1}
\input{../../modules/series/arithmetic-sum}
\input{../../modules/series/arithmetic-sum-theorem}
\input{../../modules/series/geometric-sum}
\input{../../modules/series/series-convergence}
\input{../../modules/series/series-geometric-ex1}
\input{../../modules/series/series-geometric-theorem}
\input{../../modules/series/series-geometric-ex}
\input{../../modules/series/series-geometric-ex4}
\input{../../modules/series/series-telescoping-ex6}
\input{../../modules/series/series-harmonic-ex7}
}

\lect{\semester}{Lecture 17}{17}{
%DesiredLectureName: Basic_Divergence_Test_Integral_Test_Comparison_Test
\section{Basic divergence tests}
\input{../../modules/series/series-divergence-test}
\input{../../modules/series/series-divergence-test-ex8}
\section{The Integral Test and Estimates of Sums}
\subsection{The Integral Test}
\input{../../modules/series/integral-test-intro}
\input{../../modules/series/integral-test-above}
\input{../../modules/series/integral-test-below}
\input{../../modules/series/integral-test-def}
\input{../../modules/series/integral-test-ex1}
\input{../../modules/series/integral-test-ex2}
\input{../../modules/series/p-series}
\input{../../modules/series/integral-test-ex4}
\subsection{Estimating Sums}
\input{../../modules/series/integral-test-estimate}
\input{../../modules/series/integral-test-estimate-ex5}
\input{../../modules/series/integral-test-estimate-improvement}
\section{The Comparison Test}
\input{../../modules/series/comparison-intro}
\input{../../modules/series/comparison-theorem}
\input{../../modules/series/comparison-ex1}
\input{../../modules/series/comparison-ex2}
\input{../../modules/series/limit-comparison-intro}
\input{../../modules/series/limit-comparison-theorem}
\input{../../modules/series/limit-comparison-ex3}
\input{../../modules/series/limit-comparison-ex4}
}

\lect{\semester}{Lecture 18}{18}{
%DesiredLectureName: Alternating_Series_Absolute_Convergence_Ratio_Root_Tests
\section{Alternating Series}
\input{../../modules/series/alternating-def}
\input{../../modules/series/alternating-theorem}
\input{../../modules/series/alternating-ex1}
\input{../../modules/series/alternating-ex2}
\subsection{Estimating Sums}
\input{../../modules/series/alternating-estimate}
\input{../../modules/series/alternating-estimate-ex4}
\input{../../modules/series/absolute-convergence-intro}
\subsection{Absolute Convergence}
\input{../../modules/series/absolute-convergence-def}
\input{../../modules/series/absolute-convergence-ex1}
\input{../../modules/series/absolute-convergence-ex2}
\input{../../modules/series/conditional-convergence-def}
% WARNING:  I lack a proof of the fact that absolute convergence
% implies convergence.  I should add it to this next module.
\input{../../modules/series/absolute-implies-convergence}
\input{../../modules/series/absolute-convergence-ex3}
\section{Absolute Convergence and the Ratio and Root Tests}
\subsection{The Ratio Test}
\input{../../modules/series/ratio-test}
\input{../../modules/series/ratio-test-inconclusive}
\input{../../modules/series/ratio-test-ex4}
\input{../../modules/series/ratio-test-related-to-e-as-a-limit-1}
\input{../../modules/series/ratio-test-ex5}
\subsection{The Root Test}
\input{../../modules/series/root-test}
\input{../../modules/series/root-test-ex6}
}% end lecture

% begin lecture
\lect{\semester}{Lecture 19}{19}{
%DesiredLectureName: Power_Series
\section{Power Series}
\input{../../modules/power-series/power-series-def}
\input{../../modules/power-series/power-series-ex1}
\input{../../modules/power-series/power-series-ex2}
\input{../../modules/power-series/power-series-ex3}
\input{../../modules/power-series/radius-of-convergence}
\input{../../modules/power-series/interval-of-convergence-endpoints}
\input{../../modules/power-series/power-series-ex4}
\section{Power Series as Functions}
\input{../../modules/power-series/power-series-as-function-intro}
\input{../../modules/power-series/power-series-as-function-ex1}
\input{../../modules/power-series/power-series-as-function-ex2}
\input{../../modules/power-series/power-series-as-function-ex3}
\subsection{Differentiation and Integration of Power Series}
\input{../../modules/power-series/power-series-calculus}
\input{../../modules/power-series/power-series-calculus-ex4}
\input{../../modules/power-series/power-series-calculus-ex6}
\input{../../modules/power-series/power-series-calculus-ex7}
% WARNING: I want to include some integral approximation questions here.
\section{Taylor and Maclaurin Series}
\input{../../modules/power-series/taylor-series-intro}
\input{../../modules/power-series/taylor-series-def}
\input{../../modules/power-series/maclaurin-series-def}
\input{../../modules/power-series/maclaurin-series-ex1}
%\input{../../modules/power-series/taylor-series-representation}
\input{../../modules/power-series/maclaurin-series-e-ex}
\input{../../modules/power-series/taylor-series-ex3}
\input{../../modules/power-series/maclaurin-series-ex4}
\input{../../modules/power-series/maclaurin-series-sin-ex}
\input{../../modules/power-series/maclaurin-series-ex5}
\input{../../modules/power-series/maclaurin-series-ex6}
\input{../../modules/power-series/maclaurin-series-list}
\input{../../modules/power-series/maclaurin-series-ex11}
\input{../../modules/power-series/maclaurin-series-limit-ex}
}% end lecture

\lect{\semester}{Lecture 20}{20}{
%DesiredLectureName: A_Bit_Of_Differential_Equations
\section{Modeling with Differential Equations}
\input{../../modules/diff-eq-models/diff-eq-def}
\subsection{Models of Population Growth}
\input{../../modules/diff-eq-models/diff-eq-natural-growth}
\input{../../modules/diff-eq-models/diff-eq-natural-growth-solution}
\input{../../modules/diff-eq-models/diff-eq-logistic}
\input{../../modules/diff-eq-models/diff-eq-logistic-graph}
\subsection{A Model for the Motion of a Spring}
\input{../../modules/diff-eq-models/diff-eq-spring}
\subsection{General Differential Equations}
\input{../../modules/diff-eq-models/diff-eq-terminology}
\input{../../modules/diff-eq-models/diff-eq-ex1}
\input{../../modules/diff-eq-models/diff-eq-initial-condition}
\input{../../modules/diff-eq-models/diff-eq-initial-condition-ex2}
\section{Direction Fields and Euler's Method}
\input{../../modules/diff-eq-direction-fields/direction-fields-intro}
\subsection{Direction Fields}
\input{../../modules/diff-eq-direction-fields/direction-fields-procedure}
\section{Separable Equations}
\input{../../modules/diff-eq-separable/diff-eq-separable-def}
\input{../../modules/diff-eq-separable/diff-eq-separable-solution}
\input{../../modules/diff-eq-separable/diff-eq-separable-ex1}
\input{../../modules/diff-eq-separable/diff-eq-separable-ex3}
\subsection{Orthogonal Trajectories}
\input{../../modules/diff-eq-separable/orthogonal-trajectory-def}
\input{../../modules/diff-eq-separable/orthogonal-trajectory-ex5}
\subsection{Mixing Problems}
\input{../../modules/diff-eq-separable/mixing-problem-intro}
\input{../../modules/diff-eq-separable/mixing-problem-ex6}
\section{Models for Population Growth}
\subsection{The Law of Natural Growth}
\input{../../modules/diff-eq-models/natural-growth-reminder}
\input{../../modules/diff-eq-models/natural-growth-solution}
\subsection{The Logistic Model}
\input{../../modules/diff-eq-models/logistic-reminder}
\input{../../modules/diff-eq-models/logistic-solution}
\input{../../modules/diff-eq-models/logistic-ex2}
}% end lecture

\lect{\semester}{Lecture 21}{21}{
%DesiredLectureName: Complex_Numbers
\section{Complex numbers}
\input{../../modules/complex-numbers/complex-numbers-definition}
\input{../../modules/complex-numbers/complex-numbers-terminology}
\input{../../modules/complex-numbers/complex-numbers-addition-multiplication-example-1}
\input{../../modules/complex-numbers/complex-conjugation}
\input{../../modules/complex-numbers/complex-conjugation-is-field-hmm}
\input{../../modules/complex-numbers/complex-numbers-division-example-1}
\input{../../modules/complex-numbers/complex-numbers-division}
\input{../../modules/complex-numbers/product-of-exponents-is-exponent-of-sum}
\input{../../modules/complex-numbers/complex-exponent-def}
\input{../../modules/trigonometry/trig-Euler-formula}
\input{../../modules/complex-numbers/complex-numbers-polar-form}
\input{../../modules/complex-numbers/complex-numbers-polar-form-example-1}
\input{../../modules/complex-numbers/complex-numbers-polar-form-example-2}
\input{../../modules/complex-numbers/complex-numbers-exponent-definition1-multiplication-angles-add}
\input{../../modules/complex-numbers/complex-numbers-exponent-definition2-multiplication-angles-add}
\input{../../modules/complex-numbers/complex-numbers-multiplication-arguments-add}
\input{../../modules/complex-numbers/complex-numbers-de-Moivre-formula}
\input{../../modules/complex-numbers/complex-numbers-de-Moivre-formula-example}
\input{../../modules/complex-numbers/complex-roots-example-1}
\input{../../modules/complex-numbers/complex-roots-example-2}
}

\end{document}
