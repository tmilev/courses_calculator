\documentclass
%[handout]
{beamer}
\newcommand{\semester}{2018}
\usepackage[breakwords]{truncate}
\input{../../freecalc/lectures/example-templates}
\input{../../freecalc/lectures/system-specific-config-Ubuntu-texlive} %
\input{../../freecalc/lectures/pstricks-commands}
\renewcommand{\Arcsin}{\arcsin}
\renewcommand{\Arccos}{\arccos}
\renewcommand{\Arctan}{\arctan}
\renewcommand{\Arccot}{\text{arccot\hspace{0.03cm}}}
\renewcommand{\Arcsec}{\text{arcsec\hspace{0.03cm}}}
\renewcommand{\Arccsc}{\text{arccsc\hspace{0.03cm}}}
\useinnertheme{rounded}
\useoutertheme{infolines}
\usecolortheme{orchid}
\usecolortheme{whale}


\usepackage[T1]{fontenc}
% Or whatever. Note that the encoding and the font should match. If T1
% does not look nice, try deleting the line with the fontenc.

\setbeamertemplate{navigation symbols}{}

\newcommand{\lect}[4]{
\ifnum#3=\currentLecture
  \date{#1}
  \lecture[#1]{#2}{#3}
#4
\else
%include nothing
\fi
}

\setbeamertemplate{headline}
{
  \leavevmode%
  \hbox{%
  \begin{beamercolorbox}[wd=.4\paperwidth,ht=2.25ex,dp=1ex,center]{author in head/foot}%
    \usebeamerfont{title in head/foot}{\insertsectionhead}
  \end{beamercolorbox}%
  \begin{beamercolorbox}[wd=.4\paperwidth,ht=2.25ex,dp=1ex,center]{title in head/foot}%
    \usebeamerfont{title in head/foot}{\insertsubsectionhead}
  \end{beamercolorbox}%
\begin{beamercolorbox}[wd=.2\paperwidth,ht=2.25ex,dp=1ex,center]{date in head/foot}%
\insertframenumber/\inserttotalframenumber
  \end{beamercolorbox}
}
}

\setbeamertemplate{footline}
{
  \leavevmode%
  \hbox{%
  \begin{beamercolorbox}[wd=.4\paperwidth,ht=2.25ex,dp=1ex,center]{author in head/foot}%
    \usebeamerfont{author in head/foot}\insertshortauthor
  \end{beamercolorbox}%
  \begin{beamercolorbox}[wd=.4\paperwidth,ht=2.25ex,dp=1ex,center]{title in head/foot}%
    \usebeamerfont{title in head/foot}{\insertshorttitle}
  \end{beamercolorbox}%
  \begin{beamercolorbox}[wd=.2\paperwidth,ht=2.25ex,dp=1ex,center]{date in head/foot}%
    \usebeamerfont{date in head/foot}\insertshortdate{}
  \end{beamercolorbox}}%
  \vskip0pt%
}


\AtBeginLecture{%
\title[{\truncate{4.2cm}{\insertlecture}}]{
\textbf{\color{green}{Cryptography 101}}
\\
\textbf{\insertlecture} \\
{\color{green}{\textbf{calculator-algebra.org}}}
}
\author[{\color{green}{\bf calculator-algebra.org}}, Todor Milev]{
~\\~\\~\\
\begin{tabular}{c}
\Large Todor Milev 
\end{tabular}
}


\date{\insertshortlecture}

\begin{frame}
\titlepage
\end{frame}
}

\begin{document}
\providecommand{\currentLecture}{1}


\lect{\semester}{Lecture 0}{0}{
%DesiredLectureName: NOT_COVERED_Review_Representing_Functions
\section{Ways to Represent a Function}
\subsection{The Definition of a Function}
\input{../../modules/functions-basics/function-def}
\subsection{The Vertical Line Test}
\input{../../modules/precalculus/vertical-line-test}
\subsection{Piecewise Defined Functions}
\input{../../modules/precalculus/function-piecewise}
\input{../../modules/precalculus/absolute-value}
\input{../../modules/precalculus/piecewise-formula}
\input{../../modules/precalculus/piecewise-ex1}
\input{../../modules/precalculus/piecewise-ex2}
\subsection{Symmetry}
\input{../../modules/precalculus/even-and-odd}
\subsection{Increasing and Decreasing Functions}
\input{../../modules/precalculus/increasing-decreasing}
\subsection{A Note on Domains of Functions}
\input{../../modules/functions-basics/domains}
}% end lecture

\lect{\semester}{Lecture 1}{1}{
%DesiredLectureName: Review_Basic_Functions_New_Functions_From_Old

\section{A Catalog of Essential Functions}
\subsection{Linear Functions}
\input{../../freecalc/modules/precalculus/linear-functions}
\subsection{Polynomials}
\input{../../freecalc/modules/precalculus/polynomials}
\subsection{Power Functions}
\input{../../freecalc/modules/precalculus/power-functions-def}
\input{../../freecalc/modules/precalculus/root-functions}
\input{../../freecalc/modules/precalculus/reciprocal-function}
\subsection{Rational Functions}
\input{../../freecalc/modules/precalculus/rational-functions}
\subsection{Algebraic Functions}
\input{../../freecalc/modules/precalculus/algebraic-functions}
\subsection{Transcendental Functions}
\input{../../freecalc/modules/precalculus/transcendental-functions-version2}
\section{New Functions from Old Functions}
\input{../../freecalc/modules/precalculus/combinations-functions}
\input{../../freecalc/modules/precalculus/composition-functions}
\input{../../freecalc/modules/precalculus/composition-example}
\input{../../freecalc/modules/precalculus/composition-example-2}
%\section{Composing Functions with Linear Transformations}
%\input{../../freecalc/modules/function-graph-linear-transformations/transformations-shifts}
%\input{../../freecalc/modules/function-graph-linear-transformations/transformations-shifts-example}
%\input{../../freecalc/modules/function-graph-linear-transformations/transformations-magnifications}
%\input{../../freecalc/modules/function-graph-linear-transformations/transformations-horizontal-stretches}
%\input{../../freecalc/modules/function-graph-linear-transformations/transformations-absolute-value}
}% end lecture

\lect{\semester}{Lecture 2}{2}{%begin lecture
%DesiredLectureName: Review_Trigonometry
\section{Trigonometry}
\subsection{Angles}
\input{../../modules/trigonometry/angles-measurement-units}
\input{../../modules/trigonometry/frequently-encountered-angles}
\subsection{The Trigonometric Functions}
\input{../../modules/trigonometry/trig-functions-and-right-angle-triangles}
\input{../../modules/trigonometry/trig-functions-unit-circle}
\input{../../modules/trigonometry/trig-example}
\input{../../modules/trigonometry/trig-functions-example2}
\subsection{Trigonometric Identities}
\input{../../modules/trigonometry/trig-functions-even-odd}
\input{../../modules/trigonometry/trig-functions-periodicity}
\input{../../modules/trigonometry/trig-identity-definition}
\input{../../modules/trigonometry/trig-identities-pythagorean}
\input{../../modules/trigonometry/trig-identities-angle-sum-summary}
\input{../../modules/trigonometry/trig-identities-example-1}
\input{../../modules/trigonometry/trig-identities-strategy-note}
\input{../../modules/trigonometry/trig-identities-example-2}
\section{Trigonometric equations}
\input{../../modules/trigonometry/trig-equations-intro}
\input{../../modules/trigonometry/trig-equations-example}
\input{../../modules/trigonometry/trig-equations-strategy}
\input{../../modules/trigonometry/trig-equations-example-2}
\subsection{Trigonometric Identities and Complex Numbers}
\input{../../modules/complex-numbers/complex-numbers-definition}
\youWillNotBeTested
\input{../../modules/trigonometry/trig-Euler-formula} 
\youWillNotBeTested
\input{../../modules/trigonometry/trig-using-euler-formula}
\subsection{Graphs of the Trigonometric Functions}
\input{../../modules/trigonometry/graph-sin-extended}
\input{../../modules/trigonometry/graph-cos-extended}
\input{../../modules/trigonometry/graph-tan-extended}
\input{../../modules/trigonometry/graphs-sin-and-cos}
\input{../../modules/trigonometry/graphs-tan-and-cot}
\input{../../modules/trigonometry/graphs-sec-and-csc}
}% end lecture

\lect{\semester}{Lecture 3}{3}{% begin lecture
%DesiredLectureName: Limits
\section{The Limit of a Function}
\input{../../modules/limits/limit-def}
\input{../../modules/limits/limit-ex1}
\input{../../modules/limits/limit-ex3}
\input{../../modules/limits/limit-ex4}
\subsection{One-sided Limits}
\input{../../modules/limits/limits-one-sided-ex6}
\input{../../modules/limits/limits-one-sided-def}
\input{../../modules/limits/limits-ex7}
\section{Calculating Limits Using Limit Laws}
\input{../../modules/limits/limit-laws}
\input{../../modules/limits/limit-laws-ex2}
\input{../../modules/limits/limit-laws-using-ex}
\input{../../modules/limits/direct-substitution-version2}
\input{../../modules/limits/direct-substitution-yes-ex}
\input{../../modules/limits/direct-substitution-no-ex}
\input{../../modules/limits/limit-laws-fact}
\input{../../modules/limits/limit-factoring-ex1}
\input{../../modules/limits/limit-laws-ex6}
\input{../../modules/limits/limit-laws-ex4}
\input{../../modules/limits/limit-laws-ex5}
\input{../../modules/limits/limits-piecewise}
\input{../../modules/limits/limits-piecewise-ex9}
\input{../../modules/limits/squeeze-theorem}
\input{../../modules/limits/squeeze-theorem-ex11}
}% end lecture

\lect{\semester}{Lecture 4}{4}{% begin lecture
%DesiredLectureName: Continuity
\section{Continuity}
\input{../../modules/continuity/continuous-def}
\input{../../modules/continuity/discontinuous-def-version2}
\input{../../modules/continuity/continuity-ex1}
\input{../../modules/continuity/greatest-integer-function}
\input{../../modules/continuity/continuity-ex-removable-defined}
\input{../../modules/continuity/continuity-ex-removable-infinite}
\input{../../modules/continuity/continuity-ex-removable-jump}
\input{../../modules/continuity/continuous-one-sided}
\input{../../modules/continuity/continuity-ex3}
\input{../../modules/continuity/continuity-on-interval}
\input{../../modules/continuity/continuity-laws}
\input{../../modules/continuity/continuous-functions-classes}
\input{../../modules/continuity/continuity-ex5}
\input{../../modules/continuity/continuity-ex8}
\section{Intermediate Value Theorem}
\input{../../modules/continuity/IVT}
\input{../../modules/continuity/IVT-ex9}
} %end lecture

\lect{\semester}{Lecture 5}{5}{% begin lecture
%DesiredLectureName: Limits_Involving_Infinity
\section{Limits Involving Infinity}
\subsection{Infinite Limits}
\input{../../modules/limits/limits-infinite-ex8}
\input{../../modules/limits/limits-infinite-def}
\input{../../modules/limits/vertical-asymptote-def}
\input{../../modules/limits/limits-ex9}
\input{../../modules/limits/infinite-limit-rules}
\input{../../modules/limits/infinite-limit-ex1}
\input{../../modules/limits/infinite-limit-ex2}
\input{../../modules/limits/infinite-limit-ln-tan}
\subsection{Limits at Infinity; Horizontal Asymptotes}
\input{../../modules/curve-sketching/limit-at-infinity-intro}
\input{../../modules/curve-sketching/limit-at-infinity-def}
\input{../../modules/curve-sketching/horizontal-asymptote-def}
\input{../../modules/curve-sketching/limit-at-infinity-ex2}
\input{../../modules/curve-sketching/limit-at-infinity-power-function}
\input{../../modules/curve-sketching/limit-at-infinity-ex3}
\input{../../modules/curve-sketching/limit-at-infinity-ex4}
\input{../../modules/curve-sketching/limit-at-infinity-ex5}
\subsection{Infinite Limits at Infinity}
\input{../../modules/curve-sketching/infinite-limit-at-infinity-def}
\input{../../modules/curve-sketching/infinite-limit-at-infinity-ex8}
\input{../../modules/curve-sketching/infinite-limit-at-infinity-ex9}
\input{../../modules/curve-sketching/infinite-limit-at-infinity-ex11}
}% end lecture

\lect{\semester}{Lecture 6}{6}{% begin lecture
%DesiredLectureName: Review_Inverse_Functions
\section{Inverse Functions}
\subsection{One-to-one Functions}
\input{../../modules/inverse-functions/one-to-one-def}
\input{../../modules/inverse-functions/horizontal-line-test}
\subsection{The Definition of the Inverse of $f$}
\input{../../modules/inverse-functions/inverse-function-def}
\input{../../modules/inverse-functions/inverse-notation-warning}
\input{../../modules/inverse-functions/inverse-function-equations}
\input{../../modules/inverse-functions/inverse-function-solve-for-version2}
\input{../../modules/inverse-functions/guess-and-check}
\input{../../modules/inverse-functions/inverse-function-graph}
\input{../../modules/inverse-functions/inverse-function-ex5}
\input{../../modules/inverse-functions/inverse-function-solve-for-ex1-freeCalc}
\input{../../modules/inverse-functions/inverse-function-solve-for-ex3-freeCalc}
}

\lect{\semester}{Lecture  7}{7}{% begin lecture
%DesiredLectureName: Review_Exponents_Logarithms
\section{Exponential Functions}
\input{../../modules/exponential-functions/exponential-properties}
\subsection{Two ways to define exponents}
\input{../../modules/exponential-functions/exponential-function-def-various-approaches}
\input{../../modules/exponential-functions/exponential-function-def}
\subsection{Basic properties}
\input{../../modules/exponential-functions/exponential-function-graphs-plus}
\input{../../modules/exponential-functions/exponential-versus-polynomial}
\input{../../modules/exponential-functions/exponential-function-ex-sketch}
\input{../../modules/exponential-functions/exponential-equation1}
\input{../../modules/exponential-functions/exponential-equation2}
\input{../../modules/exponential-functions/exponential-word-problem1}
\subsection{The Natural Exponential Function}
\input{../../modules/exponential-functions/natural-exponential-intro}
\section{Logarithmic Functions}
\subsection{Logarithm basics}
\input{../../modules/logarithms/logarithm-def}
\input{../../modules/logarithms/logarithm-def-ex1}
\input{../../modules/logarithms/log-and-exp}
\input{../../modules/logarithms/logarithm-graphs}
\input{../../modules/logarithms/logarithm-properties}
\input{../../modules/logarithms/logarithm-properties-ex2}
\subsection{Natural Logarithms}
\input{../../modules/logarithms/natural-logarithm-def}
\input{../../modules/logarithms/log-notation-summary}
\input{../../modules/logarithms/exponential-equation-basic-natural-base-1}
\input{../../modules/logarithms/exponential-equation-quadratic-natural-base-1}
\input{../../modules/logarithms/exponential-equation-quadratic-arbitrary-base-1}
\input{../../modules/logarithms/natural-logarithm-def-ex8}
\input{../../modules/inverse-functions/inverse-function-solve-for-ex2-freeCalc}
}% end lecture

\lect{\semester}{Lecture 8}{8}{% begin lecture
%DesiredLectureName: Derivatives_Basics
\section{Tangents}
\input{../../freecalc/modules/limits/tangent-overview}
\input{../../freecalc/modules/limits/tangent-problem}
%\input{../../freecalc/modules/limits/tangent-problem-part2}
\input{../../freecalc/modules/limits/tangent-def}
\input{../../freecalc/modules/derivatives/tangents-ex1}
\input{../../freecalc/modules/derivatives/tangents-alternative-form}
\input{../../freecalc/modules/derivatives/tangents-ex2}
\input{../../freecalc/modules/derivatives/tangent-line-polynomial}
\section{Derivatives}
\input{../../freecalc/modules/derivatives/derivative-def-version2}
\input{../../freecalc/modules/derivatives/derivatives-ex4}
\input{../../freecalc/modules/derivatives/derivatives-ex5}
\subsection{Other Notations}
\input{../../freecalc/modules/derivatives/derivative-notations}
\subsection{The Derivative as a Function}
\input{../../freecalc/modules/derivatives/derivatives-as-function}
\input{../../freecalc/modules/derivatives/derivatives-as-function-ex1}
\input{../../freecalc/modules/derivatives/derivatives-as-function-ex2}
\subsection{Velocities}
\input{../../freecalc/modules/derivatives/velocity-ex3}
\subsection{Differentiability}
\input{../../freecalc/modules/derivatives/differentiable-def}
\input{../../freecalc/modules/derivatives/differentiable-ex5}
\input{../../freecalc/modules/derivatives/differentiable-implies-continuous}
\subsection{How Can a Function Fail to be Differentiable?}
\input{../../freecalc/modules/derivatives/differentiable-counterexamples}
\subsection{Higher Derivatives}
\input{../../freecalc/modules/derivatives/higher-derivatives}
\input{../../freecalc/modules/derivatives/higher-derivatives-ex6}
\section{Differentiation Formulas}
\input{../../freecalc/modules/derivatives/differentiation-formulas-constant}
\subsection{Power Functions}
\input{../../freecalc/modules/derivatives/differentiation-formulas-square-cube}
\input{../../freecalc/modules/derivatives/differentiation-formulas-power}
\input{../../freecalc/modules/derivatives/differentiation-formulas-ex1}
\section{Balls, spheres, circles, disks and differentiation}
\youWillNotBeTested
\input{../../freecalc/modules/derivatives/differentiation-formulas-the-circle-area-perimeter-secret}
}% end lecture

\lect{\semester}{Lecture 9}{9}{% begin lecture
%DesiredLectureName: Differentiation_Product_Quotient_Rule
\section{Differentiation Formulas}
\subsection{General Power Functions}
\input{../../freecalc/modules/derivatives/arbitrary-exponents}
\input{../../freecalc/modules/derivatives/negative-exponents-ex1}
\input{../../freecalc/modules/derivatives/tangent-line-power-rule}
\input{../../freecalc/modules/derivatives/fractional-exponents-ex1}
\subsection{The Constant Multiple Rule}
\input{../../freecalc/modules/derivatives/differentiation-laws-constant}
\input{../../freecalc/modules/derivatives/constant-multiple-rule-ex1}
\input{../../freecalc/modules/derivatives/constant-multiple-rule-ex2}
\input{../../freecalc/modules/derivatives/negative-exponents-ex2}
\subsection{The Sum and Difference Rules}
\input{../../freecalc/modules/derivatives/differentiation-laws-sum}
\input{../../freecalc/modules/derivatives/differentiation-laws-difference}
\input{../../freecalc/modules/derivatives/differentiation-laws-polynomial}
\input{../../freecalc/modules/derivatives/sum-rule-with-algebra}
\subsection{Derivatives of Exponential Functions}
\input{../../freecalc/modules/exponential-functions/exponential-function-derivative}
\input{../../freecalc/modules/exponential-functions/e-def}
\input{../../freecalc/modules/exponential-functions/natural-exponential-def}
\input{../../freecalc/modules/derivatives/derivative-e-plus-polynomial}
\section{The Product and Quotient Rules}
\subsection{The Product Rule}
\input{../../freecalc/modules/product-quotient/not-product-rule}
\input{../../freecalc/modules/product-quotient/product-rule}
\input{../../freecalc/modules/product-quotient/product-rule-e-polynomial}
\subsection{The Quotient Rule}
\input{../../freecalc/modules/product-quotient/quotient-rule}
\input{../../freecalc/modules/product-quotient/quotient-rule-rational}
\input{../../freecalc/modules/product-quotient/differentiate-1-over-linear-example-1}
}% end lecture

\lect{\semester}{Lecture 10}{10}{% begin lecture
%DesiredLectureName: Trig_Derivatives
\section{Derivatives of Trigonometric Functions}
\input{../../modules/derivatives-trig/derivative-sine-graph}
\input{../../modules/derivatives-trig/derivative-sine}
\input{../../modules/derivatives-trig/derivatives-trig-ex1}
\input{../../modules/derivatives-trig/quotient-rule-e-sin}
\input{../../modules/derivatives-trig/trig-limit-ex}
\input{../../modules/derivatives-trig/derivative-cosine}
\input{../../modules/derivatives-trig/product-rule-x-cos}
\input{../../modules/derivatives-trig/derivative-tangent}
\input{../../modules/derivatives-trig/derivatives-trig-list}
\input{../../modules/derivatives-trig/derivatives-trig-ex2}
\input{../../modules/derivatives-trig/product-rule-twice-poly-e-trig}
\input{../../modules/derivatives-trig/derivatives-trig-ex4}
}% end lecture

\lect{\semester}{Lecture 11}{11}{% begin lecture
%DesiredLectureName: The_Chain_Rule
\section{The Chain Rule}
\input{../../modules/chain-rule/chain-rule-intro}
\input{../../modules/chain-rule/chain-rule-statement-extended}
\input{../../modules/chain-rule/chain-rule-notation}
\input{../../modules/chain-rule/chain-rule-ex1-Style1}
\input{../../modules/chain-rule/chain-rule-ex1-Style2}
\input{../../modules/chain-rule/chain-rule-ex1-Style3}
%\input{../../modules/chain-rule/chain-rule-sqrt-sin}
\input{../../modules/chain-rule/chain-rule-sqrt-sin-Style1}
%\input{../../modules/chain-rule/chain-rule-cos-poly}
\input{../../modules/chain-rule/chain-rule-cos-poly-Style2}
%\input{../../modules/chain-rule/chain-rule-poly-cos}
\input{../../modules/chain-rule/chain-rule-poly-cos-Style2}
\input{../../modules/chain-rule/chain-rule-power-rule}
%\input{../../modules/chain-rule/chain-rule-ex3}
\input{../../modules/chain-rule/chain-rule-ex3-Style3}
%\input{../../modules/chain-rule/chain-rule-ex4}
\input{../../modules/chain-rule/chain-rule-ex4-Style1}
\input{../../modules/chain-rule/chain-rule-ex5}
\input{../../modules/chain-rule/chain-rule-ex6}
\input{../../modules/exponential-functions/chain-rule-general-exponential-base-a-proof}
\input{../../modules/exponential-functions/general-exponential-derivative}
\input{../../modules/chain-rule/chain-rule-extra-links}
\input{../../modules/chain-rule/chain-rule-twice-ex1}
\input{../../modules/chain-rule/chain-rule-twice-ex2}
\subsection{Chain rule proof}
\input{../../modules/chain-rule/chain-rule-proof-intro}
\input{../../modules/chain-rule/chain-rule-proof}
}% end lecture

\lect{\semester}{Lecture 12}{12}{% begin lecture
%DesiredLectureName: Understanding_Derivatives
\section{Understanding computations with derivatives}
\input{../../modules/derivatives-computation-understanding/derivatives-rules-summary}
\input{../../modules/derivatives-computation-understanding/constant-multiple-rule-from-product}
\input{../../modules/derivatives-computation-understanding/power-rule-from-product-rule}
\input{../../modules/derivatives-computation-understanding/power-rule-negative-integer-from-product-rule}
\input{../../modules/derivatives-computation-understanding/power-rule-rational-from-chain-rule}
\input{../../modules/derivatives-computation-understanding/quotient-rule-from-power-and-chain}
\youWillNotBeTested
\input{../../modules/derivatives-computation-understanding/exponent-derivative-from-taylor-series}
\input{../../modules/derivatives-computation-understanding/logarithm-rule-from-exponent}
\input{../../modules/derivatives-computation-understanding/power-rule-from-exponent}
\input{../../modules/derivatives-computation-understanding/sin-cos-rules-from-exponent}
}% end lecture

\lect{\semester}{Lecture 13}{13}{% begin lecture
%DesiredLectureName: Implicit_Differentiation_Related_Rates
\section{Implicit Differentiation}
\input{../../modules/implicit-differentiation/implicit-differentiation-intro}
\input{../../modules/implicit-differentiation/implicit-tangent-line}
\input{../../modules/equation-graph/graph-equation-implicit-computer-algorithm}
\input{../../modules/implicit-differentiation/implicit-differentiation-ex3}
\input{../../modules/implicit-differentiation/implicit-differentiation-ex4}
\section{Related Rates}
\input{../../modules/related-rates/related-rates-intro}
\input{../../modules/related-rates/related-rates-ex1}
\input{../../modules/related-rates/related-rates-ex2}
%\input{../../modules/related-rates/related-rates-ex1_new}
%\input{../../modules/related-rates/related-rates-ex2_new}
%\input{../../modules/related-rates/related-rates-ex3_new}
} %end lecture

\lect{\semester}{Lecture 14}{14}{% begin lecture
%DesiredLectureName: Logarithmic_Differentiation
\section{Derivatives of Logarithmic Functions}
\input{../../freecalc/modules/logarithms/general-log-derivative-implicit}
\input{../../freecalc/modules/logarithms/general-log-derivative-ex}
\input{../../freecalc/modules/logarithms/natural-log-derivative-from-general}
\input{../../freecalc/modules/logarithms/derivative-natural-logarithm-of-linear-example-1}
\input{../../freecalc/modules/logarithms/natural-logarithm-derivative-of-radical-example-1}
\input{../../freecalc/modules/logarithms/natural-log-derivative-ex-simplify}
\input{../../freecalc/modules/logarithms/natural-logarithm-derivative-ex7}
\section{Derivative of $a(x)^{b(x)}$}
\input{../../freecalc/modules/derivatives/derivative-arbitrary-base-exponent-example-1}
\input{../../freecalc/modules/derivatives/derivative-arbitrary-base-exponent-theory-by-example}
\section{Logarithmic Differentiation}
\input{../../freecalc/modules/logarithms/logarithmic-differentiation-ex}
\input{../../freecalc/modules/logarithms/logarithmic-differentiation}
\input{../../freecalc/modules/logarithms/logarithmic-differentiation-ex-base-and-power1}
\input{../../freecalc/modules/logarithms/logarithmic-differentiation-ex-base-and-power2}
\subsection{The Number $e$ as a Limit}
\input{../../freecalc/modules/logarithms/e-limit}
}% end lecture

%Test April 9

\lect{\semester}{Lecture 15}{15}{% begin lecture
%DesiredLectureName: Maxima_Minima_Extreme_Value_Theorem_Mean_Value_Theorem
\section{Maximum and Minimum Values}
\input{../../modules/maxima-minima/max-min-intro}
\input{../../modules/maxima-minima/max-min-def}
\input{../../modules/maxima-minima/max-min-ex1}
\input{../../modules/maxima-minima/max-min-ex2}
\input{../../modules/maxima-minima/max-min-ex3}
\subsection{The Extreme Value Theorem}
\input{../../modules/maxima-minima/EVT-intro}
\input{../../modules/maxima-minima/EVT-statement}
\input{../../modules/maxima-minima/EVT-hypotheses}
\subsection{Fermat's Theorem}
\input{../../modules/maxima-minima/fermats-theorem}
\input{../../modules/maxima-minima/fermats-theorem-does-not-say}
\section{Mean Value theorem}
\input{../../modules/maxima-minima/MVT-intro}
\input{../../modules/maxima-minima/rolles-theorem}
\input{../../modules/maxima-minima/rolles-theorem-ex}
\input{../../modules/maxima-minima/MVT-meaning}
\input{../../modules/maxima-minima/MVT-proof}
\input{../../modules/maxima-minima/MVT-corollary1}
\input{../../modules/maxima-minima/MVT-corollary2}
}

\lect{\semester}{Lecture 16}{16}{% begin lecture
%DesiredLectureName: Optimization_One_Variable
\section{One Variable Optimization Problems}
\subsection{The Closed Interval Method}
\input{../../freecalc/modules/maxima-minima/critical-numbers-def}
\input{../../freecalc/modules/maxima-minima/critical-numbers-ex}
\input{../../freecalc/modules/maxima-minima/closed-interval-method}
\input{../../freecalc/modules/maxima-minima/closed-interval-method-ex}
\input{../../freecalc/modules/maxima-minima/closed-interval-method-linear-times-e-power-coeff-x-squared-1}

\subsection{Solving One Variable Optimization Problems}
\input{../../freecalc/modules/optimization/optimization-intro}
\input{../../freecalc/modules/optimization/optimization-ex1}
\input{../../freecalc/modules/optimization/optimization-ex5}
\input{../../freecalc/modules/optimization/optimization-ex1-freeCalc}
} %end lecture


\lect{\semester}{Lecture 17}{17}{% begin lecture
%DesiredLectureName: Derivatives_And_Curve_Sketching
\section{Derivatives and the Shapes of Curves}
\subsection{What Does $f'$ Say About $f$?}
\input{../../modules/curve-sketching/first-derivative-up-down}
\input{../../modules/curve-sketching/first-derivative-up-down-ex}
\input{../../modules/curve-sketching/first-derivative-test}
\subsection{What Does $f''$ Say About $f$?}
\input{../../modules/curve-sketching/concavity-def}
\input{../../modules/curve-sketching/second-derivative-concavity}
\input{../../modules/curve-sketching/inflection-point-def}
\input{../../modules/curve-sketching/second-derivative-test}
\section{Curve sketching}
\input{../../modules/curve-sketching/curve-sketching-ex6}
\input{../../modules/exponential-functions/natural-exponential-ex7}
\subsection{Curve sketching summary}
\input{../../modules/curve-sketching/curve-sketching-guidelines}
\input{../../modules/curve-sketching/curve-sketching-guidelines-ex1}
}% end lecture

\lect{\semester}{Lecture 18}{18}{% begin lecture
%DesiredLectureName: Newtons_Method
\section{Newton's Method}
\input{../../freecalc/modules/newtons-method/newtons-method-intro}
\input{../../freecalc/modules/newtons-method/newtons-method-def}
\input{../../freecalc/modules/newtons-method/newtons-method-convergence}
\input{../../freecalc/modules/newtons-method/newtons-method-ex1}
\input{../../freecalc/modules/newtons-method/newtons-method-ex-findpoly2}
} %end lecture

\lect{\semester}{Lecture 19}{19}{% begin lecture
%DesiredLectureName: Linear_Approximations_Differentials_Normals
\section{Linear Approximations} %there is no need to cover differentials earlier.
\input{../../freecalc/modules/differentials/differentials-intro}
\input{../../freecalc/modules/differentials/linearization-def-version2}
\input{../../freecalc/modules/differentials/linearization-ex1}
\input{../../freecalc/modules/differentials/differentials-ex3}
\section{Differentials}
\input{../../freecalc/modules/differentials/differential-def-version2}
\input{../../freecalc/modules/differentials/differentials-example1}
\input{../../freecalc/modules/differentials/differentials-rules}
\input{../../freecalc/modules/differentials/differentials-example2}
\section{Normals to graphs of functions}
\input{../../freecalc/modules/limits/tangent-def}
\input{../../freecalc/modules/lines-2d/normal-line}
\input{../../freecalc/modules/derivatives/normal-to-graph-of-function-def}
\input{../../freecalc/modules/derivatives/normal-to-graph-of-logarithm-example-1}
\input{../../freecalc/modules/derivatives/normal-to-graph-of-hyperbola-miscellaneous-example-1}
}% end lecture

\lect{\semester}{Lecture 20}{20}{% begin lecture
%DesiredLectureName: Areas_And_Integration
\section{Areas and Distances}
\subsection{The Area Problem}
\input{../../modules/integration/areas-intro}
\input{../../modules/integration/areas-ex1}
\input{../../modules/integration/areas-ex2}

\section{The Definite Integral}
\subsection{Review of the \(\sum\) notation}
\input{../../modules/series/notation-through-example-1}
\input{../../modules/integration/summation-notation-def}
\subsection{Riemann sums, areas and integrals}
\input{../../modules/integration/riemann-sum-def}
\input{../../modules/integration/area-def}
\input{../../modules/integration/definite-integral-def}
\input{../../modules/integration/definite-integral-negative}
\input{../../modules/integration/continuous-functions-integrable}
\subsection{Evaluating Integrals with Riemann Sums}
\input{../../modules/integration/summation-power-sum-formulas}
\input{../../modules/integration/definite-integral-ex2}
\subsection{Properties of the Definite Integral}
% WARNING: This next module could use some pictures.
\input{../../modules/integration/definite-integral-properties}
\input{../../modules/integration/definite-integral-properties-ex6}
\input{../../modules/integration/definite-integral-properties-split}
\input{../../modules/integration/definite-integral-properties-ex7}
\input{../../modules/integration/definite-integral-properties-comparison}
}% end lecture

\lect{\semester}{Lecture 21}{21}{% begin lecture
%DesiredLectureName: Antiderivatives_Fundamental_Theorem_Of_Calculus
\section{Antiderivatives}
\input{../../freecalc/modules/antiderivatives/antiderivative-def}
\input{../../freecalc/modules/antiderivatives/antiderivatives-ex}
\input{../../freecalc/modules/antiderivatives/antiderivatives-constant}
\input{../../freecalc/modules/antiderivatives/antiderivatives-ex1}
\input{../../freecalc/modules/antiderivatives/antiderivatives-formulas}
\input{../../freecalc/modules/antiderivatives/antiderivatives-ex2}
\input{../../freecalc/modules/antiderivatives/antiderivatives-ex-initial-value}
\section{Evaluating Definite Integrals}
\subsection{The Evaluation Theorem (FTC part 2)}
\input{../../freecalc/modules/integration/FTC-part2}
\input{../../freecalc/modules/integration/FTC-part2-ex5}
\input{../../freecalc/modules/integration/evaluation-bounds}
\input{../../freecalc/modules/integration/FTC-part2-ex6}
\input{../../freecalc/modules/integration/FTC-part2-ex7}
\subsection{Indefinite Integrals}
\input{../../freecalc/modules/integration/indefinite-integral-intro}
\input{../../freecalc/modules/integration/indefinite-integral-example-polynomial-1}
\input{../../freecalc/modules/integration/indefinite-integral-example-rational-monomial-1}
\input{../../freecalc/modules/integration/indefinite-integral-example-rational-monomial-using-sqrt-1}
\input{../../freecalc/modules/integration/indefinite-integral-example-reducing-to-rational-polynomial-1}
\input{../../freecalc/modules/integration/indefinite-integral-ex1}
\input{../../freecalc/modules/integration/indefinite-integral-ex2}
\input{../../freecalc/modules/integration/indefinite-integral-ex3}
\input{../../freecalc/modules/integration/indefinite-integral-ex5}
}% end lecture

\lect{\semester}{Lecture 22}{22}{% begin lecture
%DesiredLectureName: The_Substitution_Rule
\section{The Substitution Rule}
\input{../../modules/substitution-rule/substitution-rule-intro}
\input{../../modules/substitution-rule/substitution-rule-statement}
\input{../../modules/substitution-rule/substitution-rule-ex1}
\input{../../modules/substitution-rule/substitution-rule-ex2}
\input{../../modules/substitution-rule/substitution-rule-ex3}
\input{../../modules/substitution-rule/substitution-rule-ex4}
\input{../../modules/substitution-rule/substitution-rule-difficult-ex}
\subsection{Substitution rule and definite Integrals}
\input{../../modules/substitution-rule/substitution-rule-definite-integrals}
\input{../../modules/substitution-rule/substitution-rule-definite-integrals-ex6}
\input{../../modules/substitution-rule/substitution-rule-definite-integrals-ex7}
}

\lect{\semester}{Lecture 23}{23}{% begin lecture
%DesiredLectureName: The_Fundamental_Theorem_Of_Calculus_Both_Parts
\section{The Fundamental Theorem of Calculus}
\input{../../modules/integration/FTC-intro}
% WARNING: This section could use more pictures.
% It would be nice to have animated pictures of the area-so-far function. Todor: Greg, what do you mean?
\input{../../modules/integration/functions-defined-as-integrals}
\input{../../modules/integration/FTC-part1-demo}
\input{../../modules/integration/FTC-part1}
\input{../../modules/integration/FTC-part1-ex2}
\input{../../modules/integration/FTC-part1-exs}
\input{../../modules/integration/FTC-part1-ex4}
\input{../../modules/integration/FTC-both}
\input{../../modules/integration/FTC-part1-solving}
\subsection{Proof of FTC, part 1}
\input{../../modules/integration/FTC-part1-proof}
%Needs to be written:
%\input{../../modules/integration/FTC-part2-proof}
\section{The Net Change Theorem}
\input{../../modules/integration/net-change-intro}
\input{../../modules/integration/net-change-physics}
\input{../../modules/integration/net-change-ex6}
\input{../../modules/antiderivatives/rectilinear-motion-intro}
\input{../../modules/antiderivatives/rectilinear-motion-ex7}
}% end lecture

\lect{\semester}{Lecture 24}{24}{% begin lecture
%DesiredLectureName: Integration_And_Symmetry_Areas_Between_Function_Graphs
\section{Integration and symmetry}
% WARNING: This next module could use some pictures... pictures are being added...
\input{../../modules/integration/integrals-of-symmetric-functions}
\input{../../modules/integration/integrals-of-symmetric-functions-ex8}
\input{../../modules/integration/integrals-of-symmetric-functions-ex9}
\section{More About Areas}
% WARNING: The slides for section 6.1 have ugly pictures.
% They need to be brought up-to-date. In particular,
% the rectangles need black outlines and half-opaque orange shading.
\input{../../modules/area-between-curves/area-between-curves-intro}
\input{../../modules/area-between-curves/area-between-curves-def}
\input{../../modules/area-between-curves/area-between-curves-ex1}
\input{../../modules/area-between-curves/area-between-curves-ex2}
\input{../../modules/area-between-curves/area-between-curves-ex5}
}% end lecture

\lect{\semester}{Lecture 25}{25}{% begin lecture
%DesiredLectureName: Volumes_Of_Solids_Of_Revolution
\section{Volumes}
\input{../../modules/volumes/volumes-intro}
\input{../../modules/volumes/volumes-def}
\input{../../modules/volumes/volumes-ex2}
\input{../../modules/volumes/volumes-washer}
\input{../../modules/volumes/volumes-otherline}
\section{Volumes by Cylindrical Shells}
\input{../../modules/volumes/cylindrical-shells-intro}
\input{../../modules/volumes/cylindrical-shells-def-part1}
\input{../../modules/volumes/cylindrical-shells-def-part2}
\input{../../modules/volumes/cylindrical-shells-def-part3}
\input{../../modules/volumes/cylindrical-shells-ex}
\input{../../modules/volumes/cylindrical-shells-otherline}
\input{../../modules/volumes/volumes-guidelines}
}% end lecture
\end{document}
